\newcommand{\id}{60}
\newcommand{\nom}{Identification des SLCI}
\newcommand{\sequence}{02}
\newcommand{\nomsequence}{Systèmes Linéaires Continus Invariants}
\newcommand{\num}{01}
\newcommand{\type}{TP}
\newcommand{\descrip}{Modélisation d'un SLCI. Identification et modélisation des systèmes asservis du laboratoire}
\newcommand{\competences}{B2-04: Établir un modèle de connaissance par des fonctions de transfert. \\ &  B2-05: Modéliser le signal d'entrée. \\ &  B2-06: Établir un modèle de comportement à partir d'une réponse temporelle ou fréquentielle. }
\newcommand{\nbcomp}{3}
\newcommand{\systemes}{Axe Emericc}
\newcommand{\systemesnum}{56}
\newcommand{\systemessansaccent}{Axe Emericc}
\newcommand{\ilot}{4}
\newcommand{\ilotstr}{04}
\newcommand{\dossierilot}{\detokenize{Ilot_04 Axe Emericc}}
\newcommand{\imageun}{Axe_Emericc}

\newcommand{\miseenoeuvre}{\href{https://raw.githubusercontent.com/Costadoat/Sciences-Ingenieur/master/Systemes/Axe Emericc/AxeEmericc_MO/AxeEmericc_MO.pdf}{Mise en oeuvre}}
\newcommand{\experimental}{\href{https://raw.githubusercontent.com/Costadoat/Sciences-Ingenieur/master/Systemes/Axe Emericc/AxeEmericc_experimental.zip}{Relevés expérimentaux}}
\newcommand{\scilabxcos}{\href{https://raw.githubusercontent.com/Costadoat/Sciences-Ingenieur/master/Systemes/Axe Emericc/AxeEmericc_BF.zcos}{Modèle Scilab BF}}
