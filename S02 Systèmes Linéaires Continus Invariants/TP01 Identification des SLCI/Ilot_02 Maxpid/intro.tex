\newcommand{\id}{60}
\newcommand{\nom}{Identification des SLCI}
\newcommand{\sequence}{02}
\newcommand{\nomsequence}{Systèmes Linéaires Continus Invariants}
\newcommand{\num}{01}
\newcommand{\type}{TP}
\newcommand{\descrip}{Modélisation d'un SLCI. Identification et modélisation des systèmes asservis du laboratoire}
\newcommand{\competences}{B2-04: Établir un modèle de connaissance par des fonctions de transfert. \\ &  B2-05: Modéliser le signal d'entrée. \\ &  B2-06: Établir un modèle de comportement à partir d'une réponse temporelle ou fréquentielle. }
\newcommand{\nbcomp}{3}
\newcommand{\systemes}{Maxpid}
\newcommand{\systemesnum}{26}
\newcommand{\systemessansaccent}{Maxpid}
\newcommand{\ilot}{2}
\newcommand{\ilotstr}{02}
\newcommand{\dossierilot}{\detokenize{Ilot_02 Maxpid}}
\newcommand{\imageun}{Maxpid}

\newcommand{\scilabxcos}{\href{https://github.com/Costadoat/Sciences-Ingenieur/raw/master/Systemes/Maxpid/Maxpid_complet.zcos}{Modèle complet du  Maxpid}}
\newcommand{\matlabsimscape}{\href{https://github.com/Costadoat/Sciences-Ingenieur/raw/master/Systemes/Maxpid/Maxpid_Simscape.zip}{Modèle Simscape}}
\newcommand{\solidworks}{\href{https://github.com/Costadoat/Sciences-Ingenieur/raw/master/Systemes/Maxpid/Maxpid_Solidworks.zip}{Modèle Solidworks}}
\newcommand{\miseenoeuvre}{\href{https://github.com/Costadoat/Sciences-Ingenieur/raw/master/Systemes/Maxpid/Maxpid_MO/Maxpid_MO.pdf}{Mise en oeuvre}}
\newcommand{\edrawings}{\href{https://github.com/Costadoat/Sciences-Ingenieur/raw/master/Systemes/Maxpid/Maxpid.EASM}{Fichier eDrawing}}
\newcommand{\experimental}{\href{https://github.com/Costadoat/Sciences-Ingenieur/raw/master/Systemes/Maxpid/Maxpid_experimental.zip}{Analyse de résultats expérimentaux}}
\newcommand{\schemacinematique}{Maxpid_cinematique}
