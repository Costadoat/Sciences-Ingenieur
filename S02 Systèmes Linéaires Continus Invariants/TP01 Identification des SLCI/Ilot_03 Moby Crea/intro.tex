\newcommand{\id}{60}
\newcommand{\nom}{Identification des SLCI}
\newcommand{\sequence}{02}
\newcommand{\nomsequence}{Systèmes Linéaires Continus Invariants}
\newcommand{\num}{01}
\newcommand{\type}{TP}
\newcommand{\descrip}{Modélisation d'un SLCI. Identification et modélisation des systèmes asservis du laboratoire}
\newcommand{\competences}{B2-04: Établir un modèle de connaissance par des fonctions de transfert. \\ &  B2-05: Modéliser le signal d'entrée. \\ &  B2-06: Établir un modèle de comportement à partir d'une réponse temporelle ou fréquentielle. }
\newcommand{\nbcomp}{3}
\newcommand{\systemes}{Moby Crea}
\newcommand{\systemesnum}{55}
\newcommand{\systemessansaccent}{Moby Crea}
\newcommand{\ilot}{3}
\newcommand{\ilotstr}{03}
\newcommand{\dossierilot}{\detokenize{Ilot_03 Moby Crea}}
\newcommand{\imageun}{Moby_Crea}

\newcommand{\matlabsimscape}{\href{https://raw.githubusercontent.com/Costadoat/Sciences-Ingenieur/master/Systemes/Moby Crea/mobycrea_complet.zip}{Modèle Simulink complet}}
\newcommand{\matlabsimscapei}{\href{https://raw.githubusercontent.com/Costadoat/Sciences-Ingenieur/master/Systemes/Moby Crea/Mobycrea_Simscape.zip}{Modèle Simscape}}
\newcommand{\experimental}{\href{https://raw.githubusercontent.com/Costadoat/Sciences-Ingenieur/master/Systemes/Moby Crea/MobyCrea_experimental.zip}{Analyse de résultats expérimentaux}}
\newcommand{\schemacinematique}{MobyCrea_cinematique}
