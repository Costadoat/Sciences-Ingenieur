\newcommand{\id}{60}
\newcommand{\nom}{Identification des SLCI}
\newcommand{\sequence}{02}
\newcommand{\nomsequence}{Systèmes Linéaires Continus Invariants}
\newcommand{\num}{01}
\newcommand{\type}{TP}
\newcommand{\descrip}{Modélisation d'un SLCI. Identification et modélisation des systèmes asservis du laboratoire}
\newcommand{\competences}{B2-04: Établir un modèle de connaissance par des fonctions de transfert. \\ &  B2-05: Modéliser le signal d'entrée. \\ &  B2-06: Établir un modèle de comportement à partir d'une réponse temporelle ou fréquentielle. }
\newcommand{\nbcomp}{3}
\newcommand{\systemes}{ControlX}
\newcommand{\systemesnum}{91}
\newcommand{\systemessansaccent}{ControlX}
\newcommand{\ilot}{5}
\newcommand{\ilotstr}{05}
\newcommand{\dossierilot}{\detokenize{Ilot_05 ControlX}}
\newcommand{\imageun}{ControlX}

\newcommand{\miseenoeuvre}{\href{https://github.com/Costadoat/Sciences-Ingenieur/raw/master/Systemes/ControlX/ControlX_MO/ControlX_MO.pdf}{Mise en oeuvre}}
