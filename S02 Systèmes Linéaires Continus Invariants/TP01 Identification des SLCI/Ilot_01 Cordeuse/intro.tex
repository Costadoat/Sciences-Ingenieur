\newcommand{\id}{60}
\newcommand{\nom}{Identification des SLCI}
\newcommand{\sequence}{02}
\newcommand{\nomsequence}{Systèmes Linéaires Continus Invariants}
\newcommand{\num}{01}
\newcommand{\type}{TP}
\newcommand{\descrip}{Modélisation d'un SLCI. Identification et modélisation des systèmes asservis du laboratoire}
\newcommand{\competences}{B2-04: Établir un modèle de connaissance par des fonctions de transfert. \\ &  B2-05: Modéliser le signal d'entrée. \\ &  B2-06: Établir un modèle de comportement à partir d'une réponse temporelle ou fréquentielle. }
\newcommand{\nbcomp}{3}
\newcommand{\systemes}{Cordeuse}
\newcommand{\systemesnum}{48}
\newcommand{\systemessansaccent}{Cordeuse}
\newcommand{\ilot}{1}
\newcommand{\ilotstr}{01}
\newcommand{\dossierilot}{\detokenize{Ilot_01 Cordeuse}}
\newcommand{\imageun}{Cordeuse}

\newcommand{\videoavi}{\href{https://raw.githubusercontent.com/Costadoat/Sciences-Ingenieur/master/Systemes/Cordeuse/Corder_raquette_de_tennis.avi}{Comment corder une raquette de tennis}}
\newcommand{\videoavii}{\href{https://raw.githubusercontent.com/Costadoat/Sciences-Ingenieur/master/Systemes/Cordeuse/Demonstration_cordeuse.avi}{Démonstration de l'utilisation d'une cordeuse}}
