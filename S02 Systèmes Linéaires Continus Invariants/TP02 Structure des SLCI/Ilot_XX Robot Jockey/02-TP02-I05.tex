\newcommand{\lentree}{la consigne de positon en rotation de la baguette $\theta_{bc}$ ($deg$)}
\newcommand{\lasortie}{la position en rotation de la baguette $\theta_{b}$ ($deg$)}
\newcommand{\dusysteme}{du robot Jockey}
\newcommand{\lesgrandeurs}{\item la tension du moteur $u_m$ ($V$),
\item la position angulaire du moteur $\theta_m$ ($rad$),
\item la vitesse de rotation du moteur $\omega_m$ ($rad.s^{-1}$),
\item la force contre électromotrice $e$ ($V$),
\item le couple moteur $c_m$ ($N.m$),
\item le courant dans le moteur $i_m$ ($A$),
\item le couple lié aux frottements secs $C_f$ ($N.m$),
\item \lentree,
\item \lasortie.}
\newcommand{\lacorrection}{Modélisation
\begin{center}
$H(p)=\dfrac{F_c(p)}{U_m(p)}=\dfrac{\dfrac{K_m}{R_m.R_p.r}}{1+\dfrac{K_e.K_m}{R_m.K_c.R_p^2.r^2}.p+\dfrac{R_m.J}{R_m.K_c.R_p^2.r^2}.p^2}$
\end{center}}


