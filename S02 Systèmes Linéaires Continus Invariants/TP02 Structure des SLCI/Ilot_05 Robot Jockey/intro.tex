\newcommand{\id}{85}
\newcommand{\nom}{Structure des SLCI}
\newcommand{\sequence}{02}
\newcommand{\nomsequence}{Systèmes Linéaires Continus Invariants}
\newcommand{\num}{02}
\newcommand{\type}{TP}
\newcommand{\descrip}{Modélisation de la structure d'un SLCI. Boucles ouvertes et boucles fermées.}
\newcommand{\competences}{B2-07: Modéliser un système par schéma-blocs. \\ &  C1-01: Proposer une démarche permettant d'évaluer les performances des systèmes asservis. \\ &  C1-02: Proposer une démarche de réglage d'un correcteur.}
\newcommand{\nbcomp}{3}
\newcommand{\systemes}{Robot Jockey}
\newcommand{\systemessansaccent}{Robot Jockey}
\newcommand{\ilot}{5}
\newcommand{\ilotstr}{05}
\newcommand{\dossierilot}{\detokenize{Ilot_05 Robot Jockey}}
\newcommand{\imageun}{Robot_Jockey}

\newcommand{\urlsysteme}{\href{https://www.costadoat.fr/systeme/90}{Ressources système}}
\newcommand{\miseenoeuvre}{\href{https://github.com/Costadoat/Sciences-Ingenieur/raw/master/Systemes/Robot Jockey/Robot_Jockey_MO/Robot_Jockey_MO.pdf}{Mise en oeuvre}}
