\newcommand{\id}{59}
\newcommand{\nom}{Etude harmonique des SLCI}
\newcommand{\sequence}{02}
\newcommand{\nomsequence}{Systèmes Linéaires Continus Invariants}
\newcommand{\num}{03}
\newcommand{\type}{TP}
\newcommand{\descrip}{Réalisation de diagrammes de Bodes à partir des réponses harmoniques de systèmes.}
\newcommand{\competences}{A3-12: Identifier la structure d'un système asservi. \\ &  C2-01: Déterminer la réponse temporelle. \\ &  C2-02: Déterminer la réponse fréquentielle.  \\ &  C2-03: Déterminer les performances d'un système asservi. \\ &  C2-04: Mettre en œuvre une démarche de réglage d'un correcteur.}
\newcommand{\nbcomp}{5}
\newcommand{\systemes}{Maxpid}
\newcommand{\systemesnum}{26}
\newcommand{\systemessansaccent}{Maxpid}
\newcommand{\ilot}{1}
\newcommand{\ilotstr}{01}
\newcommand{\dossierilot}{\detokenize{Ilot_01 Maxpid}}
\newcommand{\imageun}{Maxpid}

\newcommand{\scilabxcos}{\href{https://raw.githubusercontent.com/Costadoat/Sciences-Ingenieur/master/Systemes/Maxpid/Maxpid_complet.zcos}{Modèle complet du  Maxpid}}
\newcommand{\matlabsimscape}{\href{https://raw.githubusercontent.com/Costadoat/Sciences-Ingenieur/master/Systemes/Maxpid/Maxpid_Simscape.zip}{Modèle Simscape}}
\newcommand{\solidworks}{\href{https://raw.githubusercontent.com/Costadoat/Sciences-Ingenieur/master/Systemes/Maxpid/Maxpid_Solidworks.zip}{Modèle Solidworks}}
\newcommand{\miseenoeuvre}{\href{https://raw.githubusercontent.com/Costadoat/Sciences-Ingenieur/master/Systemes/Maxpid/Maxpid_MO/Maxpid_MO.pdf}{Mise en oeuvre}}
\newcommand{\edrawings}{\href{https://raw.githubusercontent.com/Costadoat/Sciences-Ingenieur/master/Systemes/Maxpid/Maxpid.EASM}{Fichier eDrawing}}
\newcommand{\experimental}{\href{https://raw.githubusercontent.com/Costadoat/Sciences-Ingenieur/master/Systemes/Maxpid/Maxpid_experimental.zip}{Analyse de résultats expérimentaux}}
\newcommand{\schemacinematique}{Maxpid_cinematique}
