\newcommand{\id}{36}
\newcommand{\nom}{Etude harmonique des SLCI}
\newcommand{\sequence}{02}
\newcommand{\nomsequence}{Systèmes Linéaires Continus Invariants}
\newcommand{\num}{03}
\newcommand{\type}{TD}
\newcommand{\descrip}{Identifier les caractéristiques des SLCI en fonction de réponses temporelles et fréquentielles}
\newcommand{\competences}{A3-12: Identifier la structure d'un système asservi. \\ &  C2-01: Déterminer la réponse temporelle. \\ &  C2-02: Déterminer la réponse fréquentielle.  \\ &  C2-03: Déterminer les performances d'un système asservi. \\ &  C2-04: Mettre en œuvre une démarche de réglage d'un correcteur.}
\newcommand{\nbcomp}{5}
\newcommand{\systemes}{Amortisseur mécanique, Circuit RC, Four industriel}
\newcommand{\systemesnum}{30, 29, 31}
\newcommand{\systemessansaccent}{Amortisseur mecanique, Circuit RC, Four industriel}
\newcommand{\ilot}{3}
\newcommand{\ilotstr}{03}
\newcommand{\dossierilot}{\detokenize{Ilot_03 Amortisseur mécanique, Circuit RC, Four industriel}}
\newcommand{\imageun}{Amortisseur_mecanique}
\newcommand{\imagedeux}{Circuit_RC}
\newcommand{\imagetrois}{Four_industriel}

