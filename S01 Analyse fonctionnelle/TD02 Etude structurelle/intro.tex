\newcommand{\id}{25}
\newcommand{\nom}{Etude structurelle}
\newcommand{\sequence}{01}
\newcommand{\nomsequence}{Analyse fonctionnelle}
\newcommand{\num}{02}
\newcommand{\type}{TD}
\newcommand{\descrip}{Montrer le lien entre un système (et ses sous-systèmes) et les fonctions puis retourner le problème en montrant que pour une fonction il existe un certain nombre de solutions}
\newcommand{\competences}{A3-01: Associer les fonctions aux constituants. \\ &  A3-02: Justifier le choix des constituants dédiés aux fonctions d'un système. \\ &  A3-03: Identifier et décrire les chaines fonctionnelles du système. \\ &  A3-06: Caractériser un constituant de la chaine d'information. \\ &  D1-02: Repérer les constituants réalisant les principales fonctions des chaines fonctionnelles. \\ &  E1-05: Lire et décoder un document technique.}
\newcommand{\nbcomp}{6}
\newcommand{\systemes}{Barrage, Remplissage toners, Treuil électrique, Voiture}
\newcommand{\systemesnum}{4, 5, 6, 3}
\newcommand{\systemessansaccent}{Barrage, Remplissage toners, Treuil electrique, Voiture}
\newcommand{\ilot}{2}
\newcommand{\ilotstr}{02}
\newcommand{\dossierilot}{\detokenize{Ilot_02 Barrage, Remplissage toners, Treuil électrique, Voiture}}
\newcommand{\imageun}{Barrage}
\newcommand{\imagedeux}{Remplissage_toners}
\newcommand{\imagetrois}{Treuil_electrique}
\newcommand{\imagequatre}{Voiture}

