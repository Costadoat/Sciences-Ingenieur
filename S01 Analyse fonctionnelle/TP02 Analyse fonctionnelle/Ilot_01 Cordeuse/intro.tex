\newcommand{\id}{49}
\newcommand{\nom}{Analyse fonctionnelle}
\newcommand{\sequence}{01}
\newcommand{\nomsequence}{Analyse fonctionnelle}
\newcommand{\num}{02}
\newcommand{\type}{TP}
\newcommand{\descrip}{Analyse du contexte de l'ingénierie. Mise en place d'une structure d'étude. Découverte et mise en œuvre des systèmes}
\newcommand{\competences}{A1-01: Décrire le besoin et les exigences. \\ &  A1-02: Traduire un besoin fonctionnel en exigences. \\ &  A1-03: Définir les domaines d'application et les critères technico-économiques et environnementaux. \\ &  A1-04: Qualifier et quantifier les exigences. \\ &  A2-01: Isoler un système et justifier l'isolement. \\ &  A2-02: Définir les éléments influents du milieu extérieur. \\ &  A2-03: Identifier la nature des flux échangés traversant la frontière d'étude. \\ &  A3-01: Associer les fonctions aux constituants. \\ &  A3-02: Justifier le choix des constituants dédiés aux fonctions d'un système. \\ &  A3-03: Identifier et décrire les chaines fonctionnelles du système. \\ &  A3-04: Identifier et décrire les liens entre les chaines fonctionnelles. \\ &  E2-01: Choisir un outil de communication adapté à l'interlocuteur.}
\newcommand{\nbcomp}{12}
\newcommand{\systemes}{Cordeuse}
\newcommand{\systemesnum}{48}
\newcommand{\systemessansaccent}{Cordeuse}
\newcommand{\ilot}{1}
\newcommand{\ilotstr}{01}
\newcommand{\dossierilot}{\detokenize{Ilot_01 Cordeuse}}
\newcommand{\imageun}{Cordeuse}

\newcommand{\videoavi}{\href{https://github.com/Costadoat/Sciences-Ingenieur/raw/master/Systemes/Cordeuse/Corder_raquette_de_tennis.avi}{Comment corder une raquette de tennis}}
\newcommand{\videoavii}{\href{https://github.com/Costadoat/Sciences-Ingenieur/raw/master/Systemes/Cordeuse/Demonstration_cordeuse.avi}{Démonstration de l'utilisation d'une cordeuse}}
