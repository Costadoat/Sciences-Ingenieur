\newcommand{\id}{24}
\newcommand{\nom}{Etude d'un produit et d'un processus}
\newcommand{\sequence}{01}
\newcommand{\nomsequence}{Analyse fonctionnelle}
\newcommand{\num}{01}
\newcommand{\type}{TD}
\newcommand{\descrip}{Introduction du principe de frontière d'étude, mise en place d'exigences et manipulation des outils du langage SysMl}
\newcommand{\competences}{A1-01: Décrire le besoin et les exigences. \\ &  A1-02: Traduire un besoin fonctionnel en exigences. \\ &  A1-03: Définir les domaines d'application et les critères technico-économiques et environnementaux. \\ &  A1-04: Qualifier et quantifier les exigences. \\ &  A2-01: Isoler un système et justifier l'isolement. \\ &  A2-02: Définir les éléments influents du milieu extérieur. \\ &  A2-03: Identifier la nature des flux échangés traversant la frontière d'étude. \\ &  A3-01: Associer les fonctions aux constituants. \\ &  A3-02: Justifier le choix des constituants dédiés aux fonctions d'un système. \\ &  A3-03: Identifier et décrire les chaines fonctionnelles du système. \\ &  A3-04: Identifier et décrire les liens entre les chaines fonctionnelles.}
\newcommand{\nbcomp}{11}
\newcommand{\systemes}{Bioraphinerie, Freebox}
\newcommand{\systemesnum}{2, 1}
\newcommand{\systemessansaccent}{Bioraphinerie, Freebox}
\newcommand{\ilot}{1}
\newcommand{\ilotstr}{01}
\newcommand{\dossierilot}{\detokenize{Ilot_01 Bioraphinerie, Freebox}}
\newcommand{\imageun}{Bioraphinerie}
\newcommand{\imagedeux}{Freebox}

