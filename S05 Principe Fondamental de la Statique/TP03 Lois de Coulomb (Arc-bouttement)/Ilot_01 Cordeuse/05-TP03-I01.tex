\input{../../../headers/tpheaders.tex}

\section{Activité 1: Modélisation analytique}

Une fois que le cycle de mise sous tension de la cordeuse est effectué, il faut accrocher la corde à une pince afin de la maintenir sous tension.

Cette partie va permettre d'expliquer le phénomène de blocage de la corde.

~\

 \begin{minipage}{0.58\linewidth}
Tout d'abord une des raisons qui favorise le blocage est que la liaison glissière n'est pas parfaite et présente :
 \begin{itemize}
  \item un jeu (nécessaire au bon fonctionnement)
  \item des frottements inévitables.
 \end{itemize}
 \end{minipage}
 \hfill
  \begin{minipage}{0.4\linewidth}
   \centering\includegraphics[width=\linewidth]{img/cordeuse_arc.png}
  \end{minipage}

~\

\begin{minipage}{0.4\linewidth}
   \centering\includegraphics[width=\linewidth]{img/cordeuse_cin.png}
 \end{minipage}
 \hfill
  \begin{minipage}{0.58\linewidth}
Le modèle suivant permet de déterminer
 \begin{itemize}
  \item le problème est supposé plan,
  \item les liaisons en A et B sont ponctuelles avec frottement. (Soit $f=tan(\varphi)=0,35$ le coefficient de frottement 1-0),
  \item on néglige l'action de pesanteur sur 1.
 \end{itemize}
  \end{minipage}

\paragraph{Question 1:} Déterminer les actions mécaniques exercées en A et en B, des frottements sont à prendre en compte sur ces liaisons.

\paragraph{Question 2:} Isoler la pièce 1, faire le bilan des actions mécaniques et déterminer le système d'équations issu du P.F.S.

\paragraph{Question 3:} En se plaçant à la limite du glissement, déterminer la hauteur $h$ limite d'accrochage de la corde pour que la pince ne glisse pas.

\newpage

\section{Activité 2: Modélisation graphique}

Une fois que le cycle de mise sous tension de la cordeuse est effectué, il faut accrocher la corde à une pince afin de la maintenir sous tension.

Cette partie va permettre d'expliquer le phénomène de blocage de la corde.

~\

 \begin{minipage}{0.58\linewidth}
Tout d'abord une des raisons qui favorise le blocage est que la liaison glissière n'est pas parfaite et présente :
 \begin{itemize}
  \item un jeu (nécessaire au bon fonctionnement)
  \item des frottements inévitables.
 \end{itemize}
 \end{minipage}
 \hfill
  \begin{minipage}{0.4\linewidth}
   \centering\includegraphics[width=\linewidth]{img/cordeuse_arc.png}
  \end{minipage}

~\

\begin{minipage}{0.4\linewidth}
   \centering\includegraphics[width=\linewidth]{img/cordeuse_cin.png}
 \end{minipage}
 \hfill
  \begin{minipage}{0.58\linewidth}
Le modèle suivant permet de déterminer
 \begin{itemize}
  \item le problème est supposé plan,
  \item les liaisons en A et B sont ponctuelles avec frottement. (Soit $f=tan(\varphi)=0,35$ le coefficient de frottement 1-0),
  \item on néglige l'action de pesanteur sur 1.
 \end{itemize}
  \end{minipage}

\paragraph{Question 1:} Tracer, sur la figure suivante, les cônes de frottement au niveau des points A et B.

\paragraph{Question 2:} En se plaçant à la limite du glissement, effectuer les constructions graphiques issues de l'isolement de la pièce 1.

\newpage

\begin{center}
	\includegraphics[width=0.8	\linewidth]{img/cordeuse_cin.png}
\end{center}

\cleardoublepage

\section{Activité 3: Expérimentation}

\subsection{Présentation}

\begin{minipage}{0.55\linewidth}
Pour que les joueurs de tennis ou de badminton puissent atteindre leur meilleur niveau de jeu, il est indispensable que leurs raquettes soient correctement cordées à la tension souhaitée. En effet, de nombreux tennis-elbow sont souvent provoqués par des raquettes neuves mais mal cordées.
\end{minipage}\hfill
\begin{minipage}{0.4\linewidth}
 \centering\includegraphics[width=0.7\linewidth]{img/cordeuse.png}
\end{minipage}

~\

Les centres de compétition et les magasins spécialisés disposent de machines improprement appelées « à corder les raquettes » (ou « cordeuses » dans le texte) du type de celle qui sera étudiée aujourd'hui.

~\

\begin{minipage}{0.65\linewidth}
 Manipulation élémentaire :
\begin{enumerate}
 \item Mettre la machine sous tension (bouton à l'arrière, à droite du pupitre),
 \item La corde étant fixée d'un coté sur le capteur de force (non présent sur la machine industrielle), fixer l'autre extrémité du brin de la corde dans le mors de tirage,
\end{enumerate}
\end{minipage}\hfill
\begin{minipage}{0.3\linewidth}
 \centering\includegraphics[width=0.7\linewidth]{img/cordeuse_com.png}
\end{minipage}
\begin{enumerate}
\setcounter{enumi}{2}
 \item Programmer la tension souhaitée (30 daN) sur le pupitre,
 \item Appuyer sur le bouton poussoir (au dessus du pupitre) pour mettre en tension la corde. Pincer la corde pour avoir une image de la tension,
 \item Maintenir le brin de corde tendu à l'aide de la pince 1 et serrer la pince,
 \item Appuyer à nouveau sur le bouton poussoir pour relâcher la tension. Pincer à nouveau la corde pour avoir une image de la tension.
\end{enumerate}

\begin{center}
	\includegraphics[width=0.6\linewidth]{img/figures.pdf}
\end{center}

\paragraph{Question 1:} Proposer une procédure expérimentale afin de déterminer l'influence sur le déplacement de la pince de:
\begin{itemize}
 \item la hauteur $h$,
 \item l'effort $F$.
\end{itemize}

La valeur limite permettant le basculement est attendue ici.

\paragraph{Question 2:} Comparer les résultats de la question avec ceux de l'activité 1.



\end{document}
