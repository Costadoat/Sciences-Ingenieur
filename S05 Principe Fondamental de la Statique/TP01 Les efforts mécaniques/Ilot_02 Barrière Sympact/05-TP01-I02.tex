\input{../../../headers/tpheaders.tex}

\prob{Modéliser la loi de transmission mécanique d'un système} \\

\graphicspath{{../../../img/}}
\begin{center}
\def\svgwidth{\columnwidth}
\input{"../../../img/triptyque.pdf_tex"}
\end{center}

La démarche de l’ingénieur permet :
\begin{itemize}
 \item De vérifier les performances attendues d’un système, par évaluation de l’écart entre un cahier des charges et les réponses expérimentales (écart 1),
 \item De proposer et de valider des modèles d’un système à partir d’essais, par évaluation de l’écart entre les performances mesurées et les performances simulées (écart 2),
 \item De prévoir le comportement à partir de modélisations, par l’évaluation de l’écart entre les performances simulées et les performances attendues du cahier des charges (écart 3).
\end{itemize}

\documentsressource{Pour ce TP, vous aurez à votre disposition les documents suivants:
\begin{itemize}
 \item La \miseenoeuvre\ du système,
 \item de la procédure d'utilisation de Simscape disponible à la page \pageref{proceduresimscape},
 \item Les divers documents des \urlsysteme.
\end{itemize}}

\newpage

\section{Détermination de la loi de transmission des actions mécaniques}

L'objectif de cette partie est de déterminer les équations liant les actions mécaniques du système \systemes\ et de les comparer avec celles obtenues par simulation Matlab/Simscape.

On aura ainsi:
\begin{itemize}
 \item poids de la lisse $P$,
 \item couple moteur $C_m$.
\end{itemize}

\questioncomp{mod}{Écrire les torseurs d'action mécanique de chacune des liaisons ainsi que des actions mécaniques extérieures.}

\questioncomp{mod}{Isoler les solides}


\questioncomp{mod}{Déterminer le couple moteur $C_m$\ en fonction du poids de la lisse $P$\ et des paramètres géométriques du système, en utilisant l. Les dimensions seront mesurées sur le système afin d'effectuer l'application numérique.}

\questioncomp{res}{Compléter le modèle Simscape avec ces équation comme sur la procédure \pageref{proceduresimscape} et vérifier que les résultats correspondent.}

~\

\section{Vérification à l'aide de relevé expérimentaux}

Proposer un protocole expérimental afin d'obtenir le tracé précédent.

\questioncomp{exp}{Expliquer en quelques lignes le protocole expérimental mis en \oe uvre.}

\questioncomp{exp}{Mettre en \oe uvre ce protocole expérimental pour certaines positions du système.}

\questioncomp{exp}{Déterminer les écarts (et leurs origines) entre les résultats des la simulation et ceux issus de la partie expérimentale.}
 
\proceduresimscape

\ifdef{\public}{\end{document}}{}

\clearpage

\newpage

\section{Correction}

\begin{center}
 \includegraphics[width=0.6\linewidth]{img/Barriere_cin}
\end{center}

$\left\{T_{P\rightarrow 1}\right\}=\left\{\begin{array}{cc}
0 & \sim \\
-P & \sim \\
\sim & 0
\end{array}\right\}_G=
\left\{\begin{array}{cc}
0 & \sim \\
-P & \sim \\
\sim & (l.cos(\alpha+\theta_1)-L.cos\theta_1).P
\end{array}\right\}_B$

$\left\{T_{C_m\rightarrow 2}\right\}=\left\{\begin{array}{cc}
0 & \sim \\
0 & \sim \\
\sim & C_m
\end{array}\right\}_C=
\left\{\begin{array}{cc}
0 & \sim \\
0 & \sim \\
\sim & C_m
\end{array}\right\}_B$

$\left\{T_{0\rightarrow 1}\right\}=\left\{\begin{array}{cc}
X_{01} & \sim \\
Y_{01} & \sim \\
\sim & 0
\end{array}\right\}_A=
\left\{\begin{array}{cc}
X_{01} & \sim \\
Y_{01} & \sim \\
\sim & -l.cos(\alpha+\theta_1).Y_{01}+l.sin(\alpha+\theta_1).X_{01}
\end{array}\right\}_B$

$\left\{T_{0\rightarrow 2}\right\}=\left\{\begin{array}{cc}
X_{02} & \sim \\
Y_{02} & \sim \\
\sim & 0
\end{array}\right\}_C=
\left\{\begin{array}{cc}
X_{02} & \sim \\
Y_{02} & \sim \\
\sim & -R.cos(\theta_2).Y_{02}+R.sin(\theta_2).X_{02}
\end{array}\right\}_B$

$\left\{T_{2\rightarrow 1}\right\}=\left\{\begin{array}{cc}
0 & \sim \\
Y_{21} & \sim \\
\sim & 0
\end{array}\right\}_{B,R_1^*}=
\left\{\begin{array}{cc}
-sin(\alpha+\theta_1).Y_{21} & \sim \\
cos(\alpha+\theta_1).Y_{21} & \sim \\
\sim & 0
\end{array}\right\}_{B,R_0}$

Isoler 1

$\left\{\begin{array}{l}
X_{01}-sin(\alpha+\theta_1).Y_{21}=0 \\
-P+Y_{01}+cos(\alpha+\theta_1).Y_{21}=0 \\
(l.cos(\alpha+\theta_1)-L.cos\theta_1).P-l.cos(\alpha+\theta_1).Y_{01}+l.sin(\alpha+\theta_1).X_{01}=0
\end{array}\right.$

Isoler 2

$\left\{\begin{array}{l}
X_{02}+sin(\alpha+\theta_1).Y_{21}=0 \\
Y_{02}-cos(\alpha+\theta_1).Y_{21}=0 \\
C_m-R.cos\theta_2.Y_{02}+R.sin\theta_2.X_{02}=0
\end{array}\right.$

Donc, $Y_{21}=\frac{C_m}{R.cos(\theta_1-\theta_2+\alpha)}$

$(l.cos(\alpha+\theta_1)-L.cos\theta_1).P-l.cos(\alpha+\theta_1).(P-cos(\alpha+\theta_1).Y_{21})+l.sin(\alpha+\theta_1).sin(\alpha+\theta_1).Y_{21}=0$

$\frac{L}{l}.P.cos\theta_1=\frac{C_m}{R.cos(\theta_1-\theta_2+\alpha)}$

Donc $C_m=R.cos(\theta_1-\theta_2+\alpha).\frac{L}{l}.P.cos\theta_1$

\end{document}
