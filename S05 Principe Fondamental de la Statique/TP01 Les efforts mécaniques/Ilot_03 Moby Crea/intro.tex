\newcommand{\id}{65}
\newcommand{\nom}{Les efforts mécaniques}
\newcommand{\sequence}{05}
\newcommand{\nomsequence}{Principe Fondamental de la Statique}
\newcommand{\num}{01}
\newcommand{\type}{TP}
\newcommand{\descrip}{Principe Fondamental de la Statique. Modélisation des actions mécaniques}
\newcommand{\competences}{B2-16: Modéliser une action mécanique. \\ &  C1-05: Proposer une démarche permettant la détermination d'une action mécanique inconnue ou d'une loi de mo \\ &  C2-07: Déterminer les actions mécaniques en statique. \\ &  E2-01: Choisir un outil de communication adapté à l'interlocuteur.}
\newcommand{\nbcomp}{4}
\newcommand{\systemes}{Moby Crea}
\newcommand{\systemesnum}{55}
\newcommand{\systemessansaccent}{Moby Crea}
\newcommand{\ilot}{3}
\newcommand{\ilotstr}{03}
\newcommand{\dossierilot}{\detokenize{Ilot_03 Moby Crea}}
\newcommand{\imageun}{Moby_Crea}

\newcommand{\matlabsimscape}{\href{https://github.com/Costadoat/Sciences-Ingenieur/raw/master/Systemes/Moby Crea/mobycrea_complet.zip}{Modèle Simulink complet}}
\newcommand{\matlabsimscapei}{\href{https://github.com/Costadoat/Sciences-Ingenieur/raw/master/Systemes/Moby Crea/Mobycrea_Simscape.zip}{Modèle Simscape}}
\newcommand{\experimental}{\href{https://github.com/Costadoat/Sciences-Ingenieur/raw/master/Systemes/Moby Crea/MobyCrea_experimental.zip}{Analyse de résultats expérimentaux}}
\newcommand{\miseenoeuvre}{\href{https://github.com/Costadoat/Sciences-Ingenieur/raw/master/Systemes/Moby Crea/MobyCrea_MO/MobyCrea_MO.pdf}{Mise en oeuvre}}
\newcommand{\schemacinematique}{MobyCrea_cinematique}
