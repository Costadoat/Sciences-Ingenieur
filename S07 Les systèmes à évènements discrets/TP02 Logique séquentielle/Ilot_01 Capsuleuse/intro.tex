\newcommand{\id}{67}
\newcommand{\nom}{Logique séquentielle}
\newcommand{\sequence}{07}
\newcommand{\nomsequence}{Les systèmes à évènements discrets}
\newcommand{\num}{02}
\newcommand{\type}{TP}
\newcommand{\descrip}{Analyse du comportement algorithmique de systèmes}
\newcommand{\competences}{A3-04: Identifier et décrire les liens entre les chaines fonctionnelles. \\ &  A3-07: Analyser un algorithme.  \\ &  A3-11: Interpréter tout ou partie de l'évolution temporelle d'un système séquentiel. \\ &  B2-20: Décrire le comportement d'un système séquentiel. \\ &  E2-01: Choisir un outil de communication adapté à l'interlocuteur.}
\newcommand{\nbcomp}{5}
\newcommand{\systemes}{Capsuleuse}
\newcommand{\systemesnum}{50}
\newcommand{\systemessansaccent}{Capsuleuse}
\newcommand{\ilot}{1}
\newcommand{\ilotstr}{01}
\newcommand{\dossierilot}{\detokenize{Ilot_01 Capsuleuse}}
\newcommand{\imageun}{Capsuleuse}

\newcommand{\matlabsimscape}{\href{https://github.com/Costadoat/Sciences-Ingenieur/raw/master/Systemes/Capsuleuse/Capsuleuse_Simscape.zip}{Modèle Simscape}}
\newcommand{\solidworks}{\href{https://github.com/Costadoat/Sciences-Ingenieur/raw/master/Systemes/Capsuleuse/Capsuleuse_Solidworks.zip}{Modèles Solidworks}}
\newcommand{\miseenoeuvre}{\href{https://github.com/Costadoat/Sciences-Ingenieur/raw/master/Systemes/Capsuleuse/Capsuleuse_MO/Capsuleuse_MO.pdf}{Mise en oeuvre de la capsuleuse}}
\newcommand{\edrawings}{\href{https://github.com/Costadoat/Sciences-Ingenieur/raw/master/Systemes/Capsuleuse/Capsuleuse.EASM}{Modèle eDrawings}}
\newcommand{\experimental}{\href{https://github.com/Costadoat/Sciences-Ingenieur/raw/master/Systemes/Capsuleuse/Capsuleuse_experimental.zip}{Analyse de résultats expérimentaux}}
\newcommand{\schemacinematique}{Capsuleuse_cinematique}
