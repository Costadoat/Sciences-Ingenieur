\newcommand{\id}{134}
\newcommand{\nom}{Tolérancement et spécification}
\newcommand{\sequence}{09}
\newcommand{\nomsequence}{Conception}
\newcommand{\num}{04}
\newcommand{\type}{TP}
\newcommand{\descrip}{Tolérancement et mise en place des zones de spécifications sur des pièces de systèmes du labo}
\newcommand{\competences}{A5-04: Justifier le besoin fonctionnel d'une spécification. \\ &  A5-05: Décoder les spécifications géométriques par taille, par zone et par gabarit. \\ &  A5-06: Analyser le lien entre la liaison mécanique et les systèmes de référence associés aux surfaces des c}
\newcommand{\nbcomp}{3}
\newcommand{\systemes}{Barrière Sympact}
\newcommand{\systemesnum}{49}
\newcommand{\systemessansaccent}{Barriere Sympact}
\newcommand{\ilot}{4}
\newcommand{\ilotstr}{04}
\newcommand{\dossierilot}{\detokenize{Ilot_04 Barrière Sympact}}
\newcommand{\imageun}{Barriere}

\newcommand{\solidworks}{\href{https://github.com/Costadoat/Sciences-Ingenieur/raw/master/Systemes/Barriere Sympact/Barriere_Solidworks.zip}{Modèle Solidworks}}
\newcommand{\matlabsimscape}{\href{https://github.com/Costadoat/Sciences-Ingenieur/raw/master/Systemes/Barriere Sympact/Barriere_Simscape.zip}{Modèle Simscape}}
\newcommand{\edrawings}{\href{https://github.com/Costadoat/Sciences-Ingenieur/raw/master/Systemes/Barriere Sympact/Barriere.EASM}{Modèle eDrawings}}
\newcommand{\videoavi}{\href{https://github.com/Costadoat/Sciences-Ingenieur/raw/master/Systemes/Barriere Sympact/Utilisation_barriere_1.avi}{Utilisation de la barrière}}
\newcommand{\videoavii}{\href{https://github.com/Costadoat/Sciences-Ingenieur/raw/master/Systemes/Barriere Sympact/Utilisation_barriere_2.avi}{Utilisation de la barrière avec télécommande}}
\newcommand{\experimental}{\href{https://github.com/Costadoat/Sciences-Ingenieur/raw/master/Systemes/Barriere Sympact/Barriere_experimental.zip}{Analyse de résultats expérimentaux}}
\newcommand{\miseenoeuvre}{\href{https://github.com/Costadoat/Sciences-Ingenieur/raw/master/Systemes/Barriere Sympact/Barriere_MO/Barriere_MO.pdf}{Mise en oeuvre}}
\newcommand{\schemacinematique}{Barriere_cinematique}
