\newcommand{\id}{134}
\newcommand{\nom}{Tolérancement et spécification}
\newcommand{\sequence}{09}
\newcommand{\nomsequence}{Conception}
\newcommand{\num}{04}
\newcommand{\type}{TP}
\newcommand{\descrip}{Tolérancement et mise en place des zones de spécifications sur des pièces de systèmes du labo}
\newcommand{\competences}{A5-04: Justifier le besoin fonctionnel d'une spécification. \\ &  A5-05: Décoder les spécifications géométriques par taille, par zone et par gabarit. \\ &  A5-06: Analyser le lien entre la liaison mécanique et les systèmes de référence associés aux surfaces des c}
\newcommand{\nbcomp}{3}
\newcommand{\systemes}{Cordeuse}
\newcommand{\systemesnum}{48}
\newcommand{\systemessansaccent}{Cordeuse}
\newcommand{\ilot}{2}
\newcommand{\ilotstr}{02}
\newcommand{\dossierilot}{\detokenize{Ilot_02 Cordeuse}}
\newcommand{\imageun}{Cordeuse}

\newcommand{\videoavi}{\href{https://github.com/Costadoat/Sciences-Ingenieur/raw/master/Systemes/Cordeuse/Corder_raquette_de_tennis.avi}{Comment corder une raquette de tennis}}
\newcommand{\videoavii}{\href{https://github.com/Costadoat/Sciences-Ingenieur/raw/master/Systemes/Cordeuse/Demonstration_cordeuse.avi}{Démonstration de l'utilisation d'une cordeuse}}
