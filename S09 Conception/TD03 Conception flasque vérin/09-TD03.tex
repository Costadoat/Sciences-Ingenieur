\input{../../headers/tdheaders.tex}

\section{Vérin pneumatique}

Le vérin d'équilibrage est un vérin pneumatique. Il est composé de deux flasques (avant 6 et arrière 5), d'une tige 1 vissée sur le piston 3 qui glisse à l'intérieur du tube 2.

Le tout est maintenu serré par quatre tirants d'assemblages 4 avec écrous 10 et rondelles plates 11.

On donne les différentes pièces du vérin d'équilibrage (documents 11 et 12) sauf le
flasque avant 6 à concevoir.

\paragraph{Question 1:} Concevoir le flasque avant 6 du vérin pneumatique, à l'échelle 1:1, en respectant les consignes suivantes :
\begin{itemize}
 \item S'inspirer du flasque arrière 5 pour les épaisseurs,
 \item Le brut de cette pièce est moulé,
 \item Le guidage de la tige aura une longueur mini égale à 2 fois son diamètre,
 \item Un joint de tige 6 sera monté pour assurer l'étanchéité entre la tige 1 et le flasque 6,
 \item Prévoir l'orifice d'alimentation en air,
 \item Ne représenter qu'un seul tirant 4 en coupe partielle DD,
 \item Ne pas représenter les arêtes cachées.
\end{itemize}

\paragraph{Question 2:} Représenter le flasque 6 sur plusieurs vue à l'image du document 12.


\newpage

\includegraphics[width=0.9\linewidth]{plans/verin_vierge}

\includepdf[angle=90,pages=1-2]{plans/docs11-12}

\clearpage

\ifdef{\public}{\end{document}}{}

\newpage

\pagestyle{correction}\setcounter{section}{0}

\section{Correction}

\includegraphics[width=0.9\linewidth]{plans/verin_corrige}

\includepdf[pages=1]{plans/Flasque6.pdf}

\end{document}

