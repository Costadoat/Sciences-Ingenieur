\input{../../headers/tdheaders.tex}

\section{Variateur de vitesse}

\begin{minipage}{0.6\linewidth}
 Un variateur de vitesse est un dispositif mécanique permettant de faire varier, \textbf{de manière continue}, le rapport de démultiplication d'un moteur. Il est utilisé en remplacement d'une boîte de vitesses.
\end{minipage}
\hfill
\begin{minipage}{0.35\linewidth}
 \centering\includegraphics[width=0.9\linewidth]{img/variat}
\end{minipage}

L'étude va porter sur la pièce 5.

\paragraph{Question 1:} Représenter la pièce sur 3 vues représentatives.

\paragraph{Question 2:} Colorier les surfaces fonctionnelles de cette pièce.

\paragraph{Question 3:} Représenter cette pièce en incluant un ensemble de défauts géométriques.

\paragraph{Question 4:} Exprimer les défauts, concernant ces surfaces, qui doivent être limités afin de garantir le respect du fonctionnement du variateur.

\includepdf[scale=0.95,offset=10 -30]{img/Variateur_vitesse.pdf}

\ifdef{\public}{\end{document}}{}

\newpage

\pagestyle{correction}

\section{Correction}

\includepdf[scale=0.95,offset=10 -30]{img/Piece5/Piece5.pdf}

\includegraphics[width=0.8\linewidth]{img/Piece5/Piece5_1.png}

\includegraphics[width=0.8\linewidth]{img/Piece5/Piece5_2.png}

\includepdf[scale=0.95,offset=10 -30]{img/Piece5/Piece5_couleur.pdf}

\includepdf[scale=0.95,offset=10 -30]{img/Piece5/Piece5_defauts.pdf}

\includepdf[scale=0.95,offset=10 -30]{img/Piece5/Piece5_seule_specif.pdf}


\end{document}
