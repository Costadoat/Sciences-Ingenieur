\input{../../../headers/tpheaders.tex}

\section{Présentation de la mallette}

Un roulement mécanique permet le positionnement, la transmission des efforts et la rotation entre deux pièces par le remplacement du glissement en un roulement. Ce composant mécanique optimise le frottement et la précision de la liaison. Lorsque plusieurs roulements sont installés dans une configuration particulière entre deux pièces, on appelle cela un montage de roulements.

\begin{center}
\includegraphics[width=0.5\linewidth]{img/Roulement1.jpg}
\end{center}

\subsection{Roulements}

Les roulements peuvent être de plusieurs types :
\begin{itemize}
 \item à bille,
 \item à rouleaux,
 \item à contact oblique/radial,
 \item avec/sans cage.
\end{itemize}

\begin{center}
\includegraphics[width=0.5\linewidth]{img/Roulement2.jpg}
\end{center}


\subsection{Bâti/Alésage}

Le bâti sert de support au montage des roulements. Des rainures sont usinées afin de mettre en place des circlips. 4 trous filetés sont usinés sur les côtés afin de mettre en place des chapeaux.

\begin{center}
\includegraphics[width=0.5\linewidth]{img/Roulement3.jpg}
\end{center}

\subsection{Arbres}

Les arbres sont usinés afin de mettre en place des circlips ou des écrous. Des épaulements sont usinés afin de permettre de les utiliser comme arrêts axiaux.

\begin{center}
\includegraphics[width=0.6\linewidth]{img/Roulement4.jpg}
\end{center}

\subsection{Entretoises}

Les entretoises sont utilisées afin de maintenir une distance fixe entre les éléments. Celles de petit diamètre peuvent être mise en place sur l'arbre, celle de grand diamètre sur l'alésage.

\begin{center}
\includegraphics[width=0.5\linewidth]{img/Roulement5.jpg}
\end{center}

\subsection{Chapeaux}

Les chapeaux sont utilisés afin de permettre un arrêt axial au niveau de l'alésage. Il sont mis en place grâce aux 4 trous filetés usinés sur la bâti.

\begin{center}
\includegraphics[width=0.5\linewidth]{img/Roulement6.jpg}
\end{center}

\newpage

\subsection{Les joints}

Les joints peuvent être utilisés dans deux cas de figure. Dans un premier temps, ils peuvent être utilisés afin d'empêcher des impuretés de rentrer dans les roulements et ainsi, favoriser leur usure. Ils peuvent aussi être utilisés afin d'empêcher le lubrifiant de sortir du mécanisme.

\begin{center}
\includegraphics[width=0.5\linewidth]{img/Roulement7.jpg}
\end{center}

\subsection{Les éléments d'assemblage}

Les éléments d'assemblage sont utilisés afin de lier les éléments les uns aux autres.

Attention, la mise en place des circlips s'effectue grâce aux deux pinces. Leur déformation doit être faite avec précaution car, en s'échappant des pinces, ils risqueraient de vous blesser vous ou votre entourage.

\begin{center}
\includegraphics[width=0.5\linewidth]{img/Roulement8.jpg}
\end{center}

\newpage

\section{L'assemblage d'un montage de roulements}

En utilisant l'arbre \og circlips \fg, et seulement des circlips. Effectuer le montage de deux roulements à billes sur l'arbre, sans alésage.

\paragraph{Question 1:} Combien de circlips sont nécessaires ?

Existe-t-il encore un jeu axial ?

En utilisant l'alésage, et seulement des circlips. Effectuer le montage de deux roulements à billes sur l'alésage, sans arbre.

\paragraph{Question 2:} Combien de circlips sont nécessaires ?

Existe-t-il encore un jeu axial ?

Effectuer le montage complet de l'arbre dans l'alésage, à partir des deux
précédents.

\paragraph{Question 3:} Combien de circlips sont nécessaires ? Et pourquoi ?

Quels circlips avez-vous gardé et pourquoi ?

Existe-t-il toujours un jeu axial ?

Écrire le torseur des petits déplacements de l'arbre par rapport à l'alésage.

En utilisant l'arbre \og épaulement long \fg, un circlips, des entretoises et un écrou.
Effectuer le montage de deux roulements à billes sur l'arbre, sans alésage.

\paragraph{Question 4:} Existe-t-il encore un jeu axial ?

La rotation des roulements vous paraît-elle normale ?

En utilisant l'alésage, un circlips une entretoise et des chapeaux. Effectuer le montage de deux roulements à billes sur l'alésage, sans arbre.

\paragraph{Question 5:} Existe-t-il encore un jeu axial ?

La rotation des roulements vous paraît-elle normale ?

Effectuer le montage complet de l'arbre dans l'alésage, à partir des deux précédents.

\paragraph{Question 6:} Quels éléments ne sont plus nécessaires ? Pourquoi ?

Existe-t-il un jeu axial ?

Écrire le torseur des petits déplacements de l'arbre par rapport à l'alésage.

La rotation des roulements vous paraît-elle normale ? Pourquoi ?

Proposer une solution afin de permettre une bonne rotation des roulements.

Réaliser le montage de la figure suivante.

\begin{center}
\includegraphics[width=0.6\linewidth]{img/Roulement1.jpg}
\end{center}

\paragraph{Question 7:} Des quel type de montage peut-il s'agir ?

En réfléchissant à la fonction du système, ce choix vous paraît-il judicieux ? 

\section{Montages classiques}

Cas d'étude 1 : Se placer dans le cas d'utilisation du montage pour lequel, le bâti est pose sur le sol et un effort est exerce de bas en haut sur l'arbre.

Exceptionnellement pour ce TP, on ne parlera pas des ajustements qui constituent le serrage entre l'arbre/l'alésage et les roulements, mais plutôt le type de montage qui correspond, c'est a dire :
\begin{itemize}
 \item dans le cas d'un montage serre sur l'arbre, les éléments sont d'abord monte (et immobilises) sur l'arbre, puis cet ensemble est inséré dans l'alésage afin d'être fixe,
 \item dans le cas d'un montage serre sur l'alésage, les éléments sont d'abord monte (et immobilises) sur l'alésage, puis l'arbre est inséré dans cet ensemble afin d'être fixe.
\end{itemize}

\paragraph{Question 8:} Dans quel cas de montage se trouve l'exemple du cas d'étude 1 ?

Proposer un montage de et le représenter ci-dessous.

Cas d'étude 2 : Se placer dans le cas d'utilisation du montage pour lequel, le bâti est posé sur le sol et une charge excentrée est liée par encastrement sur l'arbre.

\paragraph{Question 9:} Dans quel cas de montage se trouve l'exemple du cas d'étude 1 ?

Proposer un montage de et le représenter ci-dessous.

En utilisant l'arbre \og épaulement court \fg, effectuer le montage de deux roulements à billes sur l'arbre, sans alésage. Ce montage devra permettre le réglage du serrage axial des roulements.

\paragraph{Question 10:} Réaliser un montage répondant à cette exigence et le représenter ci-dessous.

Ce type de montage permet de régler le jeu axial et les frottements dans les roulements.

\paragraph{Question 11:} Exprimer avec des outils adaptés le lien qui existe entre le jeu axial et les frottements dans les roulements. L'utilisation d'une courbe approximative construite à partir de la manipulation serait appréciée.

\section{Synthèse de l'étude}

A la suite de ce TP, vous devez être capable lors de la conception et de la représentation d'un montage de roulements de proposer une solution qui tient compte de tous les effets évoqués dans ce sujet.

\paragraph{Question 12:} Réaliser une système de ces effets. Cette synthèse pourra être le point de départ d'une conception de montage de roulements, une liste de paramètre à vérifier avant de démarrer la représentation.

\end{document}
