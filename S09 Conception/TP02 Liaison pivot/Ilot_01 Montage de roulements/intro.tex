\newcommand{\id}{58}
\newcommand{\nom}{Liaison pivot}
\newcommand{\sequence}{09}
\newcommand{\nomsequence}{Conception}
\newcommand{\num}{02}
\newcommand{\type}{TP}
\newcommand{\descrip}{Conception d'une liaison mécanique, guidage en rotation.Analyse des montages de liaisons pivot en terme des jeu, de serrage et réalisation des montages pour répondre à un cahier des charges}
\newcommand{\competences}{B2-12: Proposer une modélisation des liaisons avec leurs caractéristiques géométriques. \\ &  B2-13: Proposer un modèle cinématique paramétré à partir d'un système réel, d'une maquette numérique ou d'u \\ &  B2-17: Simplifier un modèle de mécanisme. \\ &  B2-18: Modifier un modèle pour le rendre isostatique. \\ &  E1-05: Lire et décoder un document technique. \\ &  E2-01: Choisir un outil de communication adapté à l'interlocuteur. \\ &  F2-01: Proposer et hiérarchiser des critères de choix. \\ &  F2-02: Choisir les composants de la chaine d'information. \\ &  F2-03: Choisir les composants de la chaine de puissance. \\ &  F2-04: Modifier la commande pour faire évoluer le comportement du système.  \\ &  F3-01: Dimensionner un composant des chaines fonctionnelles à partir d'une documentation technique. \\ &  F3-02: Concevoir et dimensionner une liaison mécanique.}
\newcommand{\nbcomp}{12}
\newcommand{\systemes}{Montage de roulements}
\newcommand{\systemesnum}{79}
\newcommand{\systemessansaccent}{Montage de roulements}
\newcommand{\ilot}{1}
\newcommand{\ilotstr}{01}
\newcommand{\dossierilot}{\detokenize{Ilot_01 Montage de roulements}}
\newcommand{\imageun}{Montage_de_roulements}

