\newcommand{\id}{62}
\newcommand{\nom}{Défauts géométriques}
\newcommand{\sequence}{09}
\newcommand{\nomsequence}{Conception}
\newcommand{\num}{03}
\newcommand{\type}{TP}
\newcommand{\descrip}{Défauts géométriques des pièces. Mesurer l'impact de défauts géométriques (écarts et jeux) sur le montage de roulements à billes}
\newcommand{\competences}{A1-01: Décrire le besoin et les exigences. \\ &  A1-02: Traduire un besoin fonctionnel en exigences. \\ &  A1-03: Définir les domaines d'application et les critères technico-économiques et environnementaux. \\ &  A1-04: Qualifier et quantifier les exigences. \\ &  A5-04: Justifier le besoin fonctionnel d'une spécification. \\ &  A5-05: Décoder les spécifications géométriques par taille, par zone et par gabarit. \\ &  A5-06: Analyser le lien entre la liaison mécanique et les systèmes de référence associés aux surfaces des c \\ &  E2-01: Choisir un outil de communication adapté à l'interlocuteur. \\ &  F3-03: Concevoir une pièce en optimisant le triptyque produit-procédés-matériaux. \\ &  G2-06: Contrôler la conformité géométrique et dimensionnelle d'un produit.}
\newcommand{\nbcomp}{10}
\newcommand{\systemes}{Montage de roulements}
\newcommand{\systemesnum}{79}
\newcommand{\systemessansaccent}{Montage de roulements}
\newcommand{\ilot}{1}
\newcommand{\ilotstr}{01}
\newcommand{\dossierilot}{\detokenize{Ilot_01 Montage de roulements}}
\newcommand{\imageun}{Montage_de_roulements}

