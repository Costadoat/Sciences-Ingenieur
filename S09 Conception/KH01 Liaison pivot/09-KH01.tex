% Paquets généraux
\documentclass[a4paper,12pt,titlepage]{article}
\usepackage[T1]{fontenc}
\usepackage[utf8]{inputenc}
\usepackage[french]{babel}
\usepackage[gen]{eurosym}
%\usepackage[dvips]{graphicx}
\usepackage{fancyhdr}
\usepackage{pdfpages} 
\usepackage{multido}
\usepackage{hyperref}
%\usepackage{textcomp}
%\usepackage{aeguill}
\usepackage{schemabloc}
\usepackage[bitstream-charter]{mathdesign}
\usepackage{pstricks}
\usepackage{helvet}

\newcommand{\id}{30}
\newcommand{\nom}{Calculs d'hyperstatisme}
\newcommand{\sequence}{04}
\newcommand{\nomsequence}{Liaisons entre les solides}
\newcommand{\num}{03}
\newcommand{\type}{TD}
\newcommand{\descrip}{En appliquant les règles de la théorie des mécanisme, déterminer le degré d'hyperstatisme de plusieurs systèmes et proposer des solutions afin de diminuer ce degré}
\newcommand{\competences}{B2-12: Proposer une modélisation des liaisons avec leurs caractéristiques géométriques. \\ &  B2-13: Proposer un modèle cinématique paramétré à partir d'un système réel, d'une maquette numérique ou d'u \\ &  B2-17: Simplifier un modèle de mécanisme. \\ &  B2-18: Modifier un modèle pour le rendre isostatique.}
\newcommand{\nbcomp}{4}
\newcommand{\systemes}{E.P.A.S, Machine d'essai de traction}
\newcommand{\systemesnum}{14, 13}
\newcommand{\systemessansaccent}{E.P.A.S, Machine d'essai de traction}
\newcommand{\ilot}{3}
\newcommand{\ilotstr}{03}
\newcommand{\dossierilot}{\detokenize{Ilot_03 E.P.A.S, Machine d'essai de traction}}
\newcommand{\imageun}{EPAS}
\newcommand{\imagedeux}{Machine_dessai_de_traction}


\newcommand{\auteurun}{Renaud Costadoat}
\newcommand{\institute}{Lycée Dorian}


\usepackage{color}
\usepackage{xcolor}
\usepackage{colortbl}
\usepackage{helvet}
\renewcommand{\familydefault}{\sfdefault}
\usepackage{amsfonts}
\usepackage{amsmath}
%\usepackage{xspace}
\usepackage{varioref}
\usepackage{tabularx}
%\usepackage{floatflt}
\usepackage{graphics}
\usepackage{wrapfig}
\usepackage{textcomp}
\usepackage{tikz}
\usepackage{wrapfig}
\usepackage{gensymb}
\usepackage[european]{circuitikz}
\usetikzlibrary{babel}
\usepackage{ifthen}
\usepackage{cancel}
\usepackage{etoolbox}
\usepackage{multirow}
%\usepackage{boxedminipage}
\definecolor{gris25}{gray}{0.75}
\definecolor{bleu}{RGB}{18,33,98}
\definecolor{bleuf}{RGB}{42,94,171}
\definecolor{bleuc}{RGB}{231,239,247}
\definecolor{rougef}{RGB}{185,18,27}
\definecolor{rougec}{RGB}{255,188,204}%255,230,231
\definecolor{vertf}{RGB}{103,126,82}
\definecolor{vertc}{RGB}{220,255,191}
\definecolor{forestgreen}{rgb}{0.13,0.54,0.13}
\definecolor{blcr}{rgb}{0.59,0.69,0.84}
\definecolor{blfr}{rgb}{0.32,0.51,0.75}
\definecolor{orfr}{rgb}{0.90,0.42,0.15}
\definecolor{orcr}{rgb}{0.90,0.65,0.50}
\definecolor{orangef}{rgb}{0.659,0.269,0.072}
\definecolor{orange}{rgb}{0.58,0.35,0.063}
\definecolor{orangec}{rgb}{0.43,0.32,0.25}
\definecolor{rcorrect}{rgb}{0.6,0,0}
\definecolor{sequence}{rgb}{0.75,0.75,0.75}
\definecolor{competences}{rgb}{0.61,0.73,0.35}
\definecolor{grisf}{HTML}{222222}
\definecolor{grisc}{HTML}{636363}
\definecolor{normal}{HTML}{4087c4}
\definecolor{info}{HTML}{5bc0de}
\definecolor{success}{RGB}{92,184,92}
\definecolor{warning}{RGB}{240,173,78}
\definecolor{danger}{RGB}{217,83,79}
\hypersetup{                    % parametrage des hyperliens
    colorlinks=true,                % colorise les liens
    breaklinks=true,                % permet les retours à la ligne pour les liens trop longs
    urlcolor= blfr,                 % couleur des hyperliens
    linkcolor= orange,                % couleur des liens internes aux documents (index, figures, tableaux, equations,...)
    citecolor= forestgreen                % couleur des liens vers les references bibliographiques
    }

% Mise en page
\pagestyle{fancy}

\setlength{\hoffset}{-18pt}

\setlength{\oddsidemargin}{0pt} 	% Marge gauche sur pages impaire2s
\setlength{\evensidemargin}{0pt} 	% Marge gauche sur pages paires
\setlength{\marginparwidth}{00pt} 	% Largeur de note dans la marge
\setlength{\headwidth}{481pt} 	 	% Largeur de la zone de tête (17cm)
\setlength{\textwidth}{481pt} 	 	% Largeu\textbf{r de la zone de texte (17cm)
\setlength{\voffset}{-18pt} 		% Bon pour DOS
\setlength{\marginparsep}{7pt}	 	% Séparation de la marge
\setlength{\topmargin}{-30pt} 		% Pas de marge en haut
\setlength{\headheight}{55pt} 		% Haut de page
\setlength{\headsep}{20pt} 		% Entre le haut de page et le texte
\setlength{\footskip}{30pt} 		% Bas de\textbf{ page + séparation
\setlength{\textheight}{700pt} 		% Hauteur de l'icone zone de texte (25cm)
\setlength\fboxrule{1 pt}
\renewcommand{\baselinestretch}{1}
\setcounter{tocdepth}{1}
\newcommand{\cadre}[2]
{\fbox{
  \begin{minipage}{#1\linewidth}
   \begin{center}
    #2\\
   \end{center}
  \end{minipage}
 }
}


\newcommand{\titre}[1]
{\begin{center}
\cadre{0.8}{\huge #1} 
\end{center}
}

\newcounter{num_quest} \setcounter{num_quest}{0}
\newcounter{num_rep} \setcounter{num_rep}{0}

\newcommand{\question}[1]{\refstepcounter{num_quest}\par
~\ \\ \textbf{Question \arabic{num_quest} : }#1\label{q\the\value{num_quest}}\par
}

\newcommand{\reponse}[1]
{\noindent
\rule{\linewidth}{.5pt}\\
\textbf{Question \refstepcounter{num_rep}\ref{q\the\value{num_rep}} :} ~\ \\
#1}

% En tête et pied de page
\lhead{\nom}
\rhead{\includegraphics[width=2cm]{../../img/logo}}
\lfoot{Renaud Costadoat}
\cfoot{Page \thepage}

\fancypagestyle{correction}{%
  \fancyhf{}
  \lhead{\colorbox{danger}{\begin{minipage}{0.65\paperwidth} \textcolor{white}{\textbf{Correction}} \end{minipage}} }
  \rhead{\includegraphics[width=2cm]{../../img/logo}}
  \lfoot{Renaud Costadoat}
  \rfoot{\colorbox{danger}{\begin{minipage}{0.6\paperwidth} \begin{flushright}\textcolor{white}{\textbf{Correction}}\end{flushright} \end{minipage}} }}

\renewcommand{\footrulewidth}{0.4pt}

\usepackage{eso-pic}
\newcommand{\BackgroundPic}{%
\put(0,0){%
\parbox[b][\paperheight]{\paperwidth}{%
\vfill
\begin{center}
\hspace{0.5cm}\vspace{0.5cm}
\includegraphics[width=\paperwidth,height=\paperheight,%
keepaspectratio]{../../img/fond3}%
\end{center}
\vfill
}}}

\newcommand{\BackgroundPicdeux}{%
\put(25,-30){%
\parbox[b][\paperheight]{\paperwidth}{%
\vfill
\begin{center}
\includegraphics[width=\paperwidth,height=\paperheight,%
keepaspectratio]{../../img/fond4}%
\end{center}
\vfill
}}}

\begin{document}

\AddToShipoutPicture{\BackgroundPicdeux}

\pagestyle{fancy}

\section{Engrenage conique hélicoïdal}

Le montage de roulement étudié ici est utilisé afin de guider en rotation un engrenage hélicoïdal conique.

\begin{figure}[!h]
	\begin{center}
	\includegraphics[width=0.6\linewidth]{img/Picture1}
		\caption{Montage de roulements 1}
		\label{fig:image1}
	\end{center}
\end{figure}

\paragraph{Question 1:} Donner le type de montage utilisé ici.

\paragraph{Question 2:} Donner le schéma architectural équivalent.

\paragraph{Question 3:} Indiquer sur le dessin les efforts qui vont être appliqués sur ce montage.

\paragraph{Question 4:} Justifier le choix de ce montage en fonction des efforts qui vont être appliqués.

\paragraph{Question 5:} Représenter sur feuille une nouvelle solution technologique pour remplacer l'écrou sur la gauche du montage. Cette solution devra être réalisée à partir d'un écrou à encoche et d'une rondelle.

\newpage

\section{Pignon d'attaque d'un pont arrière de camion}

Le montage de roulement étudié ici est utilisé afin de guider en rotation le pignon d'un pont arrière de camion.

\begin{figure}[!h]
	\begin{center}
	\includegraphics[width=0.6\linewidth]{img/Picture2}
		\caption{Montage de roulements 1}
		\label{fig:image1}
	\end{center}
\end{figure}

\paragraph{Question 1:} Donner le type de montage utilisé ici.

\paragraph{Question 2:} Donner le schéma architectural équivalent.

\paragraph{Question 3:} Indiquer sur le dessin les efforts qui vont être appliqués sur ce montage.

\paragraph{Question 4:} Justifier le choix de ce montage en fonction des efforts qui vont être appliqués.

\paragraph{Question 5:} Représenter sur feuille une nouvelle solution technologique pour remplacer l'écrou sur la gauche du montage. Cette solution devra être réalisée à partir d'un écrou à encoche et d'une rondelle.

\newpage

\section{Poulie de disque à affuter}

Le montage de roulement étudié ici est utilisé afin de guider en rotation une poulie et un disque à affuter.

\begin{figure}[!h]
	\begin{center}
	\includegraphics[width=0.6\linewidth]{img/Picture3}
		\caption{Montage de roulements 1}
		\label{fig:image1}
	\end{center}
\end{figure}

\paragraph{Question 1:} Donner le type de montage utilisé ici.

\paragraph{Question 2:} Donner le schéma architectural équivalent.

\paragraph{Question 3:} Indiquer sur le dessin les efforts qui vont être appliqués sur ce montage.

\paragraph{Question 4:} Justifier le choix de ce montage en fonction des efforts qui vont être appliqués.

\paragraph{Question 5:} Représenter sur feuille une nouvelle solution technologique permettant d'appliquer les efforts axiaux sur l'autre roulement.

\end{document}
