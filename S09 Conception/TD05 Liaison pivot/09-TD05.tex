\input{../../headers/tdheaders.tex}

\section{Pompe Delasco}

\begin{minipage}{0.45\linewidth}

Les pompes Delasco sont des pompes à tubes péristaltiques.

Leurs principaux avantages sont:
\begin{itemize}
 \item Acceptent les fluides les plus abrasifs, les plus fragiles et les plus corrosifs,
 \item Maintenance facile et peu coûteuse,
 \item Un seul composant de la pompe (tube) en contact avec le produit,
 \item Pouvoir d'aspiration élevé,
 \item Fonctionnement à sec.
\end{itemize}

\end{minipage}
\hfill
\begin{minipage}{0.45\linewidth}
 \centering\includegraphics[width=0.8\linewidth]{img/pompe_delasco.jpg}
\end{minipage}

\paragraph{Question 1:} Quels sont les types de composants qui assurent le guidage en rotation ?

\paragraph{Question 2:} S'il s'agit de roulements, quel est le montage mis en place ?

\paragraph{Question 3:} Quels sont les ajustements à définir?

\paragraph{Question 4:} Comment est gérée l'étanchéité du système?

\paragraph{Question 5:} Quel type de lubrification est utilisé?



\includepdf[offset=15 -20,landscape]{Ressources/01-Pompe_delasco.pdf}

\newpage

\section{Moto réducteur frein}

\begin{minipage}{0.45\linewidth}

Le motoredcuteur étudié est installé sur une chaîne de conditionnement de flacons.

Il est utilisé afin de mettre en mouvement un tapis roulant qui permet de déplacer les caisses de flacons à la fin de l'encaissement.

\end{minipage}
\hfill
\begin{minipage}{0.45\linewidth}
 \centering\includegraphics[width=0.8\linewidth]{img/motoreducteur_frein.jpg}
\end{minipage}

\paragraph{Question 1:} Quels sont les types de composants qui assurent le guidage en rotation ?

\paragraph{Question 2:} S'il s'agit de roulements, quel est le montage mis en place ?

\paragraph{Question 3:} Quels sont les ajustements à définir?

\paragraph{Question 4:} Comment est gérée l'étanchéité du système?

\paragraph{Question 5:} Quel type de lubrification est utilisé?



\includepdf[offset=15 -20,landscape]{Ressources/02-Moto_reducteur_frein.pdf}

\newpage

\section{Moteur rotatif à palettes}

\begin{minipage}{0.45\linewidth}

Comme tous les moteurs hydrauliques, les moteurs à palettes fournissent un mouvement rotatif, ils doivent être équipés de ressorts pour pousser les palettes en absence de force centrifuge au début de la rotation.

Avantage principal de cette technique :
\begin{itemize}
 \item excellent couple de démarrage environ 90 à 95\% du couple de calage,
 \item performance possible du fait de l'équilibrage parfait du rotor insensible au $\Delta$ de pression entre A et B en phase de démarrage.
\end{itemize}

\end{minipage}
\hfill
\begin{minipage}{0.45\linewidth}
 \centering\includegraphics[width=0.8\linewidth]{img/moteur_a_palettes.jpg}
\end{minipage}

\paragraph{Question 1:} Quels sont les types de composants qui assurent le guidage en rotation ?

\paragraph{Question 2:} S'il s'agit de roulements, quel est le montage mis en place ?

\paragraph{Question 3:} Quels sont les ajustements à définir?

\paragraph{Question 4:} Comment est gérée l'étanchéité du système?

\paragraph{Question 5:} Quel type de lubrification est utilisé?

\includepdf[offset=15 -20,landscape]{Ressources/03-Moteur_rotatif_a_palettes.pdf}

\newpage

\section{Pompe à pistons}

\begin{minipage}{0.45\linewidth}
Les pompes à pistons sont généralement des pompes à débit variable qui équipent les transmissions 
hydrostatiques. Il existe :
\begin{itemize}
 \item les pompes à barillet fixe,
 \item les pompes à barillet tournant (pistons en ligne et pompes à axe coudé).
\end{itemize}

La pompe étudiée ici est une pompe à barillet tournant à pistons en ligne de type pompe Vikers, à cylindrée variable et axe droit dont la course des pistons est commandée par un plateau à inclinaison variable.
\end{minipage}
\hfill
\begin{minipage}{0.45\linewidth}
 \centering\includegraphics[width=0.8\linewidth]{img/Piston-Variable-Pump.jpg}
\end{minipage}

\paragraph{Question 1:} Quels sont les types de composants qui assurent le guidage en rotation ?

\paragraph{Question 2:} S'il s'agit de roulements, quel est le montage mis en place ?

\paragraph{Question 3:} Quels sont les ajustements à définir?

\paragraph{Question 4:} Comment est gérée l'étanchéité du système?

\paragraph{Question 5:} Quel type de lubrification est utilisé?

\includepdf[offset=15 -20]{Ressources/04-Pompe_a_pistons_06.pdf}

\newpage

\section{Moteur à pistons radiaux}

\begin{minipage}{0.45\linewidth}
Les moteurs hydrauliques à pistons radiaux sont des moteurs lents, ils conservent un bon rendement même avec une vitesse inférieure à $1tr.min^{-1}$.

Ils permettent de transformer l'énergie hydrostatique de la pompe hydraulique en mouvement rotatif directement sous couple élevé, sans avoir besoin de réducteur mécanique.

Leur intérêt compétitif est surtout sur les grosses cylindrées pouvant dépasser $12l$ ($12 000cm^3$).
\end{minipage}
\hfill
\begin{minipage}{0.45\linewidth}
 \centering\includegraphics[width=0.8\linewidth]{img/moteur_pistons_radiaux.jpg}
\end{minipage}

\paragraph{Question 1:} Quels sont les types de composants qui assurent le guidage en rotation ?

\paragraph{Question 2:} S'il s'agit de roulements, quel est le montage mis en place ?

\paragraph{Question 3:} Quels sont les ajustements à définir?

\paragraph{Question 4:} Comment est gérée l'étanchéité du système?

\paragraph{Question 5:} Quel type de lubrification est utilisé?

\includepdf[offset=15 -20,landscape]{Ressources/05_Moteur_pistons_radiaux.pdf}

\end{document}
