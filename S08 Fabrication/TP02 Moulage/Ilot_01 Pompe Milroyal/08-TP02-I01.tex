\input{../../../headers/tdheaders.tex}

\section{Définition du modèle}

Le point de départ est la représentation du corps de pompe pour ajouter les portées de noyau. 

\textbf{Etapes de Construction}

\textbf{Copier} le fichier du corps et \textbf{donner} lui le nom : MODELE. 

\textit{Supposer par la suite que les barres d'outils nécessaires ont été activées et que la case \og Saisir la Cote \fg a été cochée.}

\begin{enumerate}
 \item Création d'un bossage
 \item \includegraphics[height=1.5cm]{img/SW-000.png}
 \includegraphics[height=1.5cm]{img/SW-001.png}
\includegraphics[height=1.5cm]{img/SW-002.png}
\end{enumerate}

En vue Perspective, \textbf{sélectionner} la face supérieure du cylindre. \textbf{Cliquer} \includegraphics[height=0.4cm]{img/SW-003.png}

\textbf{Tracer} un cercle de $\Phi 37$. 
Dans la seconde barre de menus, \textbf{développer} Fonctions, puis \textbf{cliquer} Base/Bossage extrudé. 
\textbf{Entrer} la hauteur 40. \textbf{Indiquer} la valeur de l'angle de dépouille 2 \textdegree.

\begin{enumerate}
\setcounter{enumi}{2}
 \item  Création des 2 autres bossages identiques.
  \textbf{Répétez} les opérations précédentes pour les 2 autres faces supérieures des blocs-cylindre.
 \includegraphics[height=1.5cm]{img/SW-004.png}
 \includegraphics[height=1.5cm]{img/SW-005.png}
 \item Création d'un bossage. 
\includegraphics[height=1.5cm]{img/SW-006.png}
\includegraphics[height=1.5cm]{img/SW-007.png}
\includegraphics[height=1.5cm]{img/SW-008.png}

En vue Perspective, \textbf{sélectionner} la face latérale indiquée et \textbf{cliquer} \includegraphics[height=0.4cm]{img/SW-009.png}.

\textbf{Tracer} un cercle de $\Phi 56$.
\end{enumerate}

Dans la seconde barre de menus, \textbf{développer} Fonctions, puis \textbf{cliquer} Base/Bossage extrudé. 

\textbf{Entrer} la hauteur 50. \textbf{Indiquer} la valeur de l'angle de dépouille 2\textdegree. 

\begin{enumerate}
\setcounter{enumi}{4}
 \item  Création d'un bossage
\includegraphics[height=1.5cm]{img/SW-013.png}
\includegraphics[height=1.5cm]{img/SW-014.png}
\includegraphics[height=1.5cm]{img/SW-015.png}
\end{enumerate}

En vue Perspective, \textbf{sélectionner} la face inférieure indiquée. \textbf{Cliquer} \includegraphics[height=0.4cm]{img/SW-016.png}.
  
\textbf{Tracer} un cercle de $\Phi 25$. 
Dans la seconde barre de menus, \textbf{développer} Fonctions, puis \textbf{cliquer} Base/Bossage extrudé. 
\textbf{Entrer} la hauteur 25. \textbf{Indiquer} la valeur de l'angle de dépouille 2\textdegree. 

 \begin{minipage}{0.7\linewidth}
  Il est possible de modifier la couleur en cliquant avec le bouton droit sur le nom de la pièce dans l'arborescence, puis Apparence / Couleur. 
 \end{minipage}
 \hfill
 \begin{minipage}{0.29\linewidth}
  \centering\centering\includegraphics[height=1.5cm]{img/SW-017.png}
 \end{minipage}

\newpage

\section{Définition des noyaux}

\textbf{Créer} le noyau relatif au corps de pompe.

\textbf{Utiliser} les fonctions élémentaires de création d'un objet. 

\textbf{Etapes de Construction}

\textbf{Supposer} par la suite que les barres d'outils nécessaires ont été activées et que la case \og Saisir la Cote \fg a été cochée. 
 
\textbf{Sauvegarder} de temps en temps (nom CORPS ). 
 
\textbf{Appuyer} sur les touches \og Ctrl + N \fg , puis OK pour créer une nouvelle pièce. 

\begin{enumerate}
 \item Création d'un corps de révolution.

Dans l'arbre de création, \textbf{sélectionner} Plan de Face, puis \textbf{cliquer} \includegraphics[height=0.4cm]{img/SW-018.png}.

\textbf{Tracez} le profil indiqué.
 \includegraphics[height=2cm]{img/SW-019.png}
\end{enumerate}

~\

Dans la seconde barre de menus, \textbf{développer} Fonctions, puis \textbf{cliquer} Bossage/ Base avec révolution. 
\textbf{Entrer} 360° 

\begin{enumerate}
\setcounter{enumi}{1}
 \item  Création d'un bossage. 
Dans l'arbre de création, \textbf{sélectionner} Plan de Dessus, puis \textbf{cliquer} \includegraphics[height=0.4cm]{img/SW-020.png}.

\textbf{Tracer} le cercle indiqué.
 \includegraphics[height=1.5cm]{img/SW-021.png} 
 \includegraphics[height=1.5cm]{img/SW-022.png}

Dans la seconde barre de menus, \textbf{développer} Fonctions, puis \textbf{cliquer} Base/Bossage extrudé.

\textbf{Entrer} la hauteur 122. (Borgne)
\end{enumerate}

\begin{enumerate}
\setcounter{enumi}{2}
 \item  \textbf{Sélectionner} la face supérieure du cylindre nouvellement créé, puis \textbf{cliquer} \includegraphics[height=0.4cm]{img/SW-023.png}.

\textbf{Tracer} le cercle indiqué.
\includegraphics[height=1.5cm]{img/SW-024.png} 
\includegraphics[height=1.5cm]{img/SW-025.png}
\includegraphics[height=1.5cm]{img/SW-026.png}

Dans la seconde barre de menus, \textbf{développer} Fonctions, puis  \textbf{cliquer} Base/Bossage extrudé. 

 \textbf{Entrer} les valeurs indiquées. 

 \item Création d'un bossage

Dans l'arbre de création, \textbf{sélectionner} Plan de Dessus, puis \textbf{cliquer} \includegraphics[height=0.4cm]{img/SW-027.png}.

\textbf{Tracer} le cercle indiqué. 
 \includegraphics[height=1.5cm]{img/SW-028.png} 
\end{enumerate}

\begin{enumerate}
\setcounter{enumi}{4}
 \item  Dans la seconde barre de menus, \textbf{développer} Fonctions, puis \textbf{cliquer} Base/Bossage extrudé. 

\textbf{Entrer} la hauteur 86. (Borgne) 

\textbf{Sélectionner} la face inférieure du cylindre nouvellement créé, puis \textbf{cliquer} \includegraphics[height=0.4cm]{img/SW-029.png}

\textbf{Tracer} le cercle indiqué. 
\includegraphics[height=1.5cm]{img/SW-030.png} 
\includegraphics[height=1.5cm]{img/SW-031.png}
\includegraphics[height=1.5cm]{img/SW-032.png}

Dans la seconde barre de menus, \textbf{développer} Fonctions, puis  \textbf{cliquer} Base/Bossage extrudé. 

 \textbf{Entrer} les valeurs indiquées. 

\end{enumerate}


\begin{enumerate}
\setcounter{enumi}{5}
 \item   Création des 2 derniers bossages. 

Une solution revient à les créer par copie. 

Dans la seconde barre de menus, \textbf{développer} Fonctions, puis \textbf{cliquer} Répétition circulaire. 
\textbf{Se placer} en perspective pour sélectionner les éléments. 
 \includegraphics[height=1.5cm]{img/SW-033.png} 
\end{enumerate}

\begin{minipage}{0.7\linewidth}
Vous pouvez modifier la couleur en cliquant avec le bouton droit sur le nom de la pièce dans 
l'arborescence, puis Apparence / Couleur 
\end{minipage}
\hfill
\begin{minipage}{0.29\linewidth}
 \includegraphics[height=1.5cm]{img/SW-034.png}
\end{minipage}

\newpage

\section{Définition du moule}
Z
\textbf{Créer} le moule relatif au corps de pompe. 

\textbf{Utiliser} les fonctions élémentaires de création d'un objet. 

\textbf{Etapes de Construction}

\textbf{Supposer} par la suite que les barres d'outils nécessaires ont été activées et que la case \og Saisir la Cote \fg a été cochée. 
 
\textbf{Sauvegarder} de temps en temps (nom CORPS ). 
 
\textbf{Appuyer} sur les touches \og Ctrl + N \fg , puis OK pour créer une nouvelle pièce. 

\begin{enumerate}
 \item  Création du conteneur.

Dans l'arbre de création, \textbf{sélectionner} Plan de Face, puis \textbf{cliquer} \includegraphics[height=0.4cm]{img/SW-035.png}.

\textbf{Tracez} un carré de 300x300.
\end{enumerate}

\begin{minipage}{0.29\linewidth} 
 \includegraphics[width=0.5\linewidth]{img/SW-036.png}
\end{minipage}
\hfill
\begin{minipage}{0.7\linewidth}
Il faut tenir compte du modèle qui va être placé.

Le plan de face va servir de plan de joint, mais il n'est pas un plan de symétrie.

\textbf{Entrer} ce qui est indiqué. \textbf{Observer} le résultat en vue de droite par exemple.

Afin de placer correctement le modèle dans le moule, \textbf{donner} une certaine transparence au moule. 

Avec le bouton droit de la souris, \textbf{cliquer} dans l'arborescence sur Boite, puis Apparence / Couleur. \textbf{Entrer} 0,80.

\includegraphics[height=1cm]{img/SW-037.png}
\end{minipage}
 
 ~\
 
Pour assurer la coïncidence des plans de joint lors de l'assemblage, \textbf{créer} un plan dans le plan de face.  

\textbf{Cliquer} dans la seconde barre de menus Géométrie de Référence / Plan.  

\textbf{Sélectionner} le plan de face et \textbf{indiquer} une distance 0mm (cela peut paraître redondant, mais en fait cela facilitera la sélection lors des contraintes d'assemblage). 

\textbf{Sauvegarder} le dessin : nommez le Boite par exemple.

\begin{enumerate}
 \setcounter{enumi}{1}
 \item   Création du moule. 

\textbf{Ouvrir} les dessins Corpsmodele et Boite. \textbf{Les réduire} en cliquant sur l'icône de réduction en haut de chaque fenêtre.  

\textbf{Appuyer} sur les touches \og Ctrl + N \fg, puis OK pour créer un assemblage temporaire. 

\textbf{Insérer} d'abord la Boite, puis CorpsModele en cliquant dans la seconde barre de menus sur \og Insérer des Composants \fg.

\textbf{Placer} nettement CorpsModele en dehors de la Boite. 

Il faut maintenant positionner le modèle dans le moule en indiquant la coïncidence des plans respectifs.

Dans la seconde barre de menus, \textbf{cliquer} Contraintes. Dans l'arborescence, \textbf{sélectionner} les Plan1 de Boite et de Corpsmodele. 
(On assurera ainsi la coïncidence des plans de joint)

\begin{minipage}{0.59\linewidth}
 \textbf{Se placer} en vue de face, \textbf{déplacer} ensuite Corpsmodele dans Boite. 
 (On ne peut pas déplacer le premier objet inséré qui sert de référence) 
\end{minipage}
\hfill
\begin{minipage}{0.4\linewidth}
 \centering\centering\includegraphics[height=1.5cm]{img/SW-038.png} 
\end{minipage} 
\end{enumerate}

\begin{enumerate}
 \setcounter{enumi}{2}
 \item Dans l'arborescence, avec le bouton droit, \textbf{cliquer} Boite, puis \textbf{Editer} la pièce. 

Dans la barre de menus, \textbf{cliquer} Insertion / Fonctions / Empreinte \includegraphics[height=0.4cm]{img/SW-039.png} ou Insertion / Moules / Empreinte. 

\begin{minipage}{0.6\linewidth}
 \textbf{Sélectionner} Corpsmodele dans l'arborescence et \textbf{indiquer} un facteur de retrait de 2\%. 
(En fait, celà dépend du matériau utilisé) 
\end{minipage}
\hfill
\begin{minipage}{0.39\linewidth}
 \includegraphics[height=1.5cm]{img/SW-040.png}  
\end{minipage}

 \item Création des demi-moules. 

\begin{minipage}{0.6\linewidth}
 \textbf{Sélectionner} plan1,  \textbf{cliquer} Couper la pièce. 
 
 Dans la barre de menus, \textbf{cliquer} Insertion / Moules / Fractionner \includegraphics[height=0.4cm]{img/SW-041.png}.
\end{minipage}
\hfill
\begin{minipage}{0.39\linewidth}
 \includegraphics[height=1.5cm]{img/SW-042.png}
\end{minipage}
\end{enumerate}

\begin{enumerate}
 \setcounter{enumi}{4}
 
\item Enregistrer les pièces

\begin{minipage}{0.7\linewidth}
  Il faut maintenant enregistrer les pièces obtenues, \textbf{cocher} les cases indiquées.
 
 Renommer les moules \og mouleINF \fg  et \og mouleSUP \fg. 
\end{minipage}
\hfill
\begin{minipage}{0.29\linewidth}
 \includegraphics[height=1.5cm]{img/SW-043.png}
\end{minipage}

 \item Assemblage final. 

\begin{minipage}{0.64\linewidth}
Le travail va consister maintenant à créer un nouvel assemblage dans lequel il faudra placer, dans l'ordre : 
\begin{itemize}
 \item le moule inférieur,
 \item le noyau,
 \item le moule supérieur. 
\end{itemize}

Il est évident qu'une réalisation complète nécessite de prévoir le chemin de coulée, les évents, les masselottes d'alimentation. 
\end{minipage}
\hfill
\begin{minipage}{0.35\linewidth}
 \includegraphics[width=0.8\linewidth]{img/SW-044.png}
\end{minipage}
\end{enumerate}


\end{document}
