%%%%%%%%%%%%
% Définition des vecteurs 
%%%%%%%%%%%%
\newcommand{\vect}[1]{\overrightarrow{#1}}
\newcommand{\axe}[2]{\left(#1,\vect{#2}\right)}
\newcommand{\couple}[2]{\left(#1,\vect{#2}\right)}
\newcommand{\angl}[2]{\left(\vect{#1},\vect{#2}\right)}

\newcommand{\rep}[1]{\mathcal{R}_{#1}}
\newcommand{\quadruplet}[4]{\left(#1;#2,#3,#4 \right)}
\newcommand{\repere}[4]{\left(#1;\vect{#2},\vect{#3},\vect{#4} \right)}
\newcommand{\base}[3]{\left(\vect{#1},\vect{#2},\vect{#3} \right)}


\newcommand{\vx}[1]{\vect{x_{#1}}}
\newcommand{\vy}[1]{\vect{y_{#1}}}
\newcommand{\vz}[1]{\vect{z_{#1}}}

% d droit pour le calcul différentiel
\newcommand{\dd}{\text{d}}

\newcommand{\inertie}[2]{I\left({#1}, #2\right)}
\newcommand{\matinertie}[7]{
\begin{pmatrix}
#1 & #6 & #5 \\
#6 & #2 & #4 \\
#5 & #4 & #3 \\
\end{pmatrix}_{#7}}
%%%%%%%%%%%%
% Définition des torseurs 
%%%%%%%%%%%%

\newcommand{\ec}[2]{%
\mathcal{E}_{c\;#1/#2}}

\newcommand{\pext}[3]{%
\mathcal{P}_{#1\rightarrow#2/#3}}

\newcommand{\pint}[3]{%
\mathcal{P}_{#1 \stackrel{\text{#3}}{\leftrightarrow} #2 }}


 \newcommand{\torseur}[1]{%
\left\{{#1}\right\}
}

\newcommand{\torseurcin}[3]{%
\left\{\mathcal{#1} \left(#2/#3 \right) \right\}
}

\newcommand{\torseurci}[2]{%
\left\{\sigma \left(#1/#2 \right) \right\}
}
\newcommand{\torseurdyn}[2]{%
\left\{\mathcal{D} \left(#1/#2 \right) \right\}
}


\newcommand{\torseurstat}[3]{%
\left\{\mathcal{#1} \left(#2\rightarrow #3 \right) \right\}
}


 \newcommand{\torseurc}[8]{%
%\left\{#1 \right\}=
\left\{
{#1}
\right\}
 = 
\left\{%
\begin{array}{cc}%
{#2} & {#5}\\%
{#3} & {#6}\\%
{#4} & {#7}\\%
\end{array}%
\right\}_{#8}%
}

 \newcommand{\torseurcol}[7]{
\left\{%
\begin{array}{cc}%
{#1} & {#4}\\%
{#2} & {#5}\\%
{#3} & {#6}\\%
\end{array}%
\right\}_{#7}%
}

 \newcommand{\torseurl}[3]{%
%\left\{\mathcal{#1}\right\}_{#2}=%
\left\{%
\begin{array}{l}%
{#1} \\%
{#2} %
\end{array}%
\right\}_{#3}%
}

% Vecteur vitesse
 \newcommand{\vectv}[3]{%
\vect{V_{{#2}/{#3}}}\left( {#1}\right)
}

% Vecteur force
\newcommand{\vectf}[2]{%
\vect{R_{ {#1} \rightarrow {#2}}}
}

% Vecteur moment stat
\newcommand{\vectm}[3]{%
\vect{\mathcal{M}_{{#2} \rightarrow {#3}}}    \left( {#1}\right)
}




% Vecteur résultante cin
\newcommand{\vectrc}[2]{%
\vect{R_{c\;{{#1}/ {#2}}}}
}
% Vecteur moment cin
\newcommand{\vectmc}[3]{%
\vect{\sigma_{{#2}/{#3}}}\left( {#1}\right)
}


% Vecteur résultante dyn
\newcommand{\vectrd}[2]{%
\vect{R_{d\;{{#1}/ {#2}}}}
}
% Vecteur moment dyn
\newcommand{\vectmd}[3]{%
\vect{\delta_{{#2}/{#3}}}\left( {#1}\right)
}

% Vecteur accélération
 \newcommand{\vectg}[3]{%
\vect{\Gamma_{{#2}/{#3}}}\left( {#1}\right)
}

% Vecteur omega
 \newcommand{\vecto}[2]{%
\vect{\Omega_{{#1}/{#2}}}
}
% }$$\left\{\mathcal{#1} \right\}_{#2} =%
% \left\{%
% \begin{array}{c}%
%  #3 \\%
%  #4 %
% \end{array}%
% \right\}_{#5}}


%% 
% Boite pour Python
\usepackage{listingsutf8}
\lstset{
	language={Python},
	columns=flexible,
	basicstyle=\ttfamily,
	keywordstyle=\color{blue}, % je voudrais mettre les mots-clé en gras (\bfseries) mais c'est incompatible avec typewriter
	commentstyle=\color{gray}, % commentaires
	%backgroundcolor=\color{green!10}, % couleur du fond
	backgroundcolor=\color{white}, % couleur du fond
	%frame=single, rulecolor=\color{green!10}, % encadrement + couleur de l'encadrement
	frame=single, rulecolor=\color{white}, % encadrement + couleur de l'encadrement
	%numbers=left, numberstyle=\tiny, stepnumber=1, numbersep=5pt,
	showstringspaces=false, % supprime les espaces apparents dans les lignes de texte
	stringstyle=\color{red!50}\itshape,
	inputencoding=utf8,
	extendedchars=true,
   literate=
  {á}{{\'a}}1 {é}{{\'e}}1 {í}{{\'i}}1 {ó}{{\'o}}1 {ú}{{\'u}}1
  {Á}{{\'A}}1 {É}{{\'E}}1 {Í}{{\'I}}1 {Ó}{{\'O}}1 {Ú}{{\'U}}1
  {à}{{\`a}}1 {è}{{\`e}}1 {ì}{{\`i}}1 {ò}{{\`o}}1 {ù}{{\`u}}1
  {À}{{\`A}}1 {È}{{\'E}}1 {Ì}{{\`I}}1 {Ò}{{\`O}}1 {Ù}{{\`U}}1
  {ä}{{\"a}}1 {ë}{{\"e}}1 {ï}{{\"i}}1 {ö}{{\"o}}1 {ü}{{\"u}}1
  {Ä}{{\"A}}1 {Ë}{{\"E}}1 {Ï}{{\"I}}1 {Ö}{{\"O}}1 {Ü}{{\"U}}1
  {â}{{\^a}}1 {ê}{{\^e}}1 {î}{{\^i}}1 {ô}{{\^o}}1 {û}{{\^u}}1
  {Â}{{\^A}}1 {Ê}{{\^E}}1 {Î}{{\^I}}1 {Ô}{{\^O}}1 {Û}{{\^U}}1
  {œ}{{\oe}}1 {Œ}{{\OE}}1 {æ}{{\ae}}1 {Æ}{{\AE}}1 {ß}{{\ss}}1
  {ű}{{\H{u}}}1 {Ű}{{\H{U}}}1 {ő}{{\H{o}}}1 {Ő}{{\H{O}}}1
  {ç}{{\c c}}1 {Ç}{{\c C}}1 {ø}{{\o}}1 {å}{{\r a}}1 {Å}{{\r A}}1
  {€}{{\EUR}}1 {£}{{\pounds}}1
}