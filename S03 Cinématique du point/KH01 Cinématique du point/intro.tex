\newcommand{\id}{72}
\newcommand{\nom}{Cinématique du point}
\newcommand{\sequence}{03}
\newcommand{\nomsequence}{Cinématique du point}
\newcommand{\num}{01}
\newcommand{\type}{KH}
\newcommand{\descrip}{Dérivation vectorielle, accélération, cinématique graphique}
\newcommand{\competences}{B2-14: Modéliser la cinématique d'un ensemble de solides. \\ &  C1-04: Proposer une démarche permettant d'obtenir une loi entrée-sortie géométrique.  \\ &  C2-05: Caractériser le mouvement d'un repère par rapport à un autre repère. \\ &  C2-06: Déterminer les relations entre les grandeurs géométriques ou cinématiques. }
\newcommand{\nbcomp}{4}
\newcommand{\systemes}{}
\newcommand{\systemesnum}{}
\newcommand{\systemessansaccent}{}
\newcommand{\ilot}{1}
\newcommand{\ilotstr}{01}
\newcommand{\dossierilot}{\detokenize{Ilot_01 }}

