\input{../../../headers/tpheaders.tex}

\prob{Modéliser la loi d'entrée/sortie cinématique d'un système} \\

\graphicspath{{../../../img/}}
\begin{center}
\def\svgwidth{\columnwidth}
\input{"../../../img/triptyque.pdf_tex"}
\end{center}

La démarche de l’ingénieur permet :
\begin{itemize}
 \item De vérifier les performances attendues d’un système, par évaluation de l’écart entre un cahier des charges et les réponses expérimentales (écart 1),
 \item De proposer et de valider des modèles d’un système à partir d’essais, par évaluation de l’écart entre les performances mesurées et les performances simulées (écart 2),
 \item De prévoir le comportement à partir de modélisations, par l’évaluation de l’écart entre les performances simulées et les performances attendues du cahier des charges (écart 3).
\end{itemize}

\documentsressource{Pour ce TP, vous aurez à votre disposition les documents suivants:
\begin{itemize}
 \item La \miseenoeuvre\ du système,
 \item de la procédure d'utilisation de Simscape disponible à la page \pageref{proceduresimscape},
 \item Les divers documents des \urlsysteme.
\end{itemize}}


\newpage

\section{Détermination de la loi d'entrée/sortie géométrique}

L'objectif de cette partie est de déterminer les équations liant les paramètres géométriques du système \systemes et de les comparer avec celles obtenues par simulation Matlab/Simscape.

\graphicspath{{"../../../Systemes/\systemessansaccent/"}}

\begin{center}
\def\svgwidth{\columnwidth}
\input{"../../../Systemes/\systemessansaccent/\schemacinematique.pdf_tex"}
\end{center}

\questioncomp{mod}{Déterminer $\theta_2$\ en fonction de $\theta_1$\ et des dimensions géométriques du système en utilisant la loi de fermeture géométrique. Les dimensions seront mesurées sur le système.}

\questioncomp{res}{Compléter le modèle Simscape avec ces équation comme sur la procédure \pageref{proceduresimscape} et vérifier que les résultats correspondent.}

\questioncomp{res}{A l'aide d'un script python, faire varier $\theta_1$\ de $0$ à $\frac{\pi}{2}$. Et tracer $\theta_2$.}

\questioncomp{exp}{Proposer un protocole permettant de mesurer les valeurs extrêmes (qui correspondent à la variation de $\theta_1$\ de $0$ à $\frac{\pi}{2}$) de $\theta_2$.}

\questioncomp{ana}{Vérifier que le résultat de la question 2 correspond à celui de la question 3.}

\newpage

\section{Détermination de la loi d'entrée/sortie cinématique}

L'objectif de cette partie est de déterminer les équations liant les paramètres cinématiques du système \systemes et de les comparer avec celles obtenues par simulation Matlab/Simscape.

On aura ainsi:
\begin{itemize}
 \item $\omega_1=\dot{\theta_1}$,
 \item $\omega_2=\dot{\theta_2}$.
\end{itemize}

\questioncomp{mod}{Déterminer $\omega_2$\ en fonction de $\omega_1$\ et des paramètres géométriques du système, en utilisant la loi de fermeture cinématique. Les dimensions seront mesurées sur le système afin d'effectuer l'application numérique.}

\questioncomp{res}{Compléter le modèle Simscape avec ces équation comme sur la procédure \pageref{proceduresimscape} et vérifier que les résultats correspondent.}

~\

L'objectif est d'obtenir le profil suivant pour la vitesse de rotation $\omega_1$.

\begin{center}
\includegraphics[width=0.8\textwidth]{img/Barriere_profil.pdf}
\end{center}


Données: $t_1=2s$, $t_2=8s$, $t_3=10s$. 
\questioncomp{mod}{Déterminer $\omega_{max}$ afin d'obtenir la variation de $\theta_1$\ de $0$ à $\frac{\pi}{2}$.}

\questioncomp{mod}{A l'aide d'un script python, déterminer le profil de vitesse à imposer à $\omega_m$.}

\newpage

\section{Vérification à l'aide de relevé expérimentaux}

Le fichier contient des relevés expérimentaux issus du système réel.


\questioncomp{exp}{Ouvrir l'ensemble des fichiers présents dans le dossier compressé et analyser leur contenu.}

\questioncomp{exp}{Expliquer en quelques lignes le protocole expérimental mis en \oe uvre.}

\questioncomp{exp}{Déterminer les écarts (et leurs origines) entre les résultats des la simulation (parties 1 et 2) et ceux issus de la partie expérimentale.}
 
\section{Préparation d'une présentation}

\questioncomp{com}{Préparer une présentation à l'aide de quelques slides pour présenter votre travail.}


\proceduresimscape

\ifdef{\public}{\end{document}}{}

\clearpage

\newpage

\section{Correction}

\subsection{Fermeture géométrique}


$\overrightarrow{OB}+\overrightarrow{BC}+\overrightarrow{CA}+\overrightarrow{AO}=\overrightarrow{0}$

$\left\{\begin{array}{l}
b+l_1.cos\theta_1-l(t).cos\theta_3=0 \\
l_1.sin\theta_1-l(t).sin\theta_3-a=0
\end{array}\right.$

$l(t)=l_0+\frac{p.\theta_m}{2.\pi}$

$\left\{\begin{array}{l}
l(t).cos\theta_3=b+l_1.cos\theta_1 \\
l(t).sin\theta_3=l_1.sin\theta_1-a
\end{array}\right.$

Donc,$l(t)=\sqrt{(b+l_1.cos\theta_1)^2+(l_1.sin\theta_1-a)^2}$ et $\theta_m=(l(t)-l_0).\frac{2.\pi}{p}$.

Et $\theta_3=\frac{b+l.cos\theta_1}{l(t)}$

\subsection{Fermeture cinématique}

$\left\{V_{1/0}\right\}=
\left\{\begin{array}{cc}
0 & 0\\
0 & 0\\
\omega_b & 0
\end{array}
\right\}_B=
\left\{
\begin{array}{cc}
0 & 0\\
0 & l_1.\omega_b\\
\omega_b & 0
\end{array}
\right\}_{C,R_1}=
\left\{
\begin{array}{cc}
0 & -sin(\theta_1).l_1.\omega_b\\
0 & cos(\theta_1).l_1.\omega_b\\
\omega_b & 0
\end{array}
\right\}_{C,R_0}$

$\left\{V_{1/0}\right\}=
\left\{
\begin{array}{cc}
0 & \frac{p*\omega_m}{2.\pi}\\
0 & 0\\
0 & 0
\end{array}
\right\}_{C,R_3}=
\left\{
\begin{array}{cc}
0 & cos(\theta_3).\frac{p*\omega_m}{2.\pi}\\
0 & sin(\theta_3).\frac{p*\omega_m}{2.\pi}\\
0 & 0
\end{array}
\right\}_{C,R_0}$

Donc, 
$\left\{
\begin{array}{l}
-sin(\theta_1).l_1.\omega_b=cos(\theta_3).\frac{p*\omega_m}{2.\pi}\\
cos(\theta_1).l_1.\omega_b=sin(\theta_3).\frac{p*\omega_m}{2.\pi}
\end{array}\right.$

Donc, $\omega_m=\frac{2.\pi}{p}.\frac{-sin(\theta_1).l_1}{cos(\theta_3)}.\omega_b$

\end{document}
