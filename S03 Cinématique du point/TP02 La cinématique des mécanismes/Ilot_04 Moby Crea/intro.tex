\newcommand{\id}{57}
\newcommand{\nom}{La cinématique des mécanismes}
\newcommand{\sequence}{03}
\newcommand{\nomsequence}{Cinématique du point}
\newcommand{\num}{02}
\newcommand{\type}{TP}
\newcommand{\descrip}{Lois E/S de fermeture géométrique et cinématique. Simulation du comportement de modèles. Proposer des lois de commande en fonction d'exigences. Présenter les modèles acausaux}
\newcommand{\competences}{B2-10: Déterminer les caractéristiques d'un solide ou d'un ensemble de solides indéformables. \\ &  B2-11: Intégrer ou modifier une pièce dans un assemblage à l'aide d'un modeleur volumique 3D. \\ &  B2-14: Modéliser la cinématique d'un ensemble de solides. \\ &  C1-04: Proposer une démarche permettant d'obtenir une loi entrée-sortie géométrique.  \\ &  C2-05: Caractériser le mouvement d'un repère par rapport à un autre repère. \\ &  C2-06: Déterminer les relations entre les grandeurs géométriques ou cinématiques.  \\ &  C3-01: Mener une simulation numérique.  \\ &  C3-02: Résoudre numériquement une équation ou un système d'équations.  \\ &  E1-05: Lire et décoder un document technique. \\ &  E2-01: Choisir un outil de communication adapté à l'interlocuteur.}
\newcommand{\nbcomp}{10}
\newcommand{\systemes}{Moby Crea}
\newcommand{\systemesnum}{55}
\newcommand{\systemessansaccent}{Moby Crea}
\newcommand{\ilot}{4}
\newcommand{\ilotstr}{04}
\newcommand{\dossierilot}{\detokenize{Ilot_04 Moby Crea}}
\newcommand{\imageun}{Moby_Crea}

\newcommand{\matlabsimscape}{\href{https://raw.githubusercontent.com/Costadoat/Sciences-Ingenieur/master/Systemes/Moby Crea/Mobycrea_Simscape.zip}{Modèle Simscape}}
\newcommand{\experimental}{\href{https://raw.githubusercontent.com/Costadoat/Sciences-Ingenieur/master/Systemes/Moby Crea/MobyCrea_experimental.zip}{Analyse de résultats expérimentaux}}
\newcommand{\miseenoeuvre}{\href{https://raw.githubusercontent.com/Costadoat/Sciences-Ingenieur/master/Systemes/Moby Crea/MobyCrea_MO/MobyCrea_MO.pdf}{Mise en oeuvre}}
\newcommand{\scilabxcos}{\href{https://raw.githubusercontent.com/Costadoat/Sciences-Ingenieur/master/Systemes/Moby Crea/MobyCrea.zcos}{Modèle Scilab}}
\newcommand{\schemacinematique}{MobyCrea_cinematique}
