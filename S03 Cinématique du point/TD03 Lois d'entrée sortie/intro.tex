\newcommand{\id}{33}
\newcommand{\nom}{Lois d'entrée sortie}
\newcommand{\sequence}{03}
\newcommand{\nomsequence}{Cinématique du point}
\newcommand{\num}{03}
\newcommand{\type}{TD}
\newcommand{\descrip}{Résoudre des exercices de cinématique en utilisant les fermetures géométriques et cinématiques}
\newcommand{\competences}{C1-04: Proposer une démarche permettant d'obtenir une loi entrée-sortie géométrique.  \\ &  C2-05: Caractériser le mouvement d'un repère par rapport à un autre repère. \\ &  C2-06: Déterminer les relations entre les grandeurs géométriques ou cinématiques. }
\newcommand{\nbcomp}{3}
\newcommand{\systemes}{Banc d'épreuve hydraulique, Chaudière à bois, Décocheuse industrielle, Moteur à explosion}
\newcommand{\systemesnum}{19, 20, 18, 10}
\newcommand{\systemessansaccent}{Banc d'epreuve hydraulique, Chaudiere a bois, Decocheuse industrielle, Moteur a explosion}
\newcommand{\ilot}{3}
\newcommand{\ilotstr}{03}
\newcommand{\dossierilot}{\detokenize{Ilot_03 Banc d'épreuve hydraulique, Chaudière à bois, Décocheuse industrielle, Moteur à explosion}}
\newcommand{\imageun}{Banc_depreuve_hydraulique}
\newcommand{\imagedeux}{Chaudiere_a_bois}
\newcommand{\imagetrois}{Decocheuse_industrielle}
\newcommand{\imagequatre}{Moteur_a_explosion}

