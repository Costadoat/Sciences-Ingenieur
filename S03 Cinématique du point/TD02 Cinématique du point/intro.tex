\newcommand{\id}{32}
\newcommand{\nom}{Cinématique du point}
\newcommand{\sequence}{03}
\newcommand{\nomsequence}{Cinématique du point}
\newcommand{\num}{02}
\newcommand{\type}{TD}
\newcommand{\descrip}{Modéliser la cinématique du mouvement de solide en utilisant des vecteurs vitesse}
\newcommand{\competences}{B2-14: Modéliser la cinématique d'un ensemble de solides. \\ &  C1-04: Proposer une démarche permettant d'obtenir une loi entrée-sortie géométrique.  \\ &  C2-05: Caractériser le mouvement d'un repère par rapport à un autre repère. \\ &  C2-06: Déterminer les relations entre les grandeurs géométriques ou cinématiques. }
\newcommand{\nbcomp}{4}
\newcommand{\systemes}{Grue à tour, Poste de palettisation}
\newcommand{\systemesnum}{16, 17}
\newcommand{\systemessansaccent}{Grue a tour, Poste de palettisation}
\newcommand{\ilot}{2}
\newcommand{\ilotstr}{02}
\newcommand{\dossierilot}{\detokenize{Ilot_02 Grue à tour, Poste de palettisation}}
\newcommand{\imageun}{Grue_a_tour}
\newcommand{\imagedeux}{Poste_de_palettisation}

