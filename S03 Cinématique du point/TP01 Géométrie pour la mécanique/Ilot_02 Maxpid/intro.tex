\newcommand{\id}{148}
\newcommand{\nom}{Géométrie pour la mécanique}
\newcommand{\sequence}{03}
\newcommand{\nomsequence}{Cinématique du point}
\newcommand{\num}{01}
\newcommand{\type}{TP}
\newcommand{\descrip}{Déterminer une fermeture géométrique et vérifier expérimentalement.}
\newcommand{\competences}{B2-14: Modéliser la cinématique d'un ensemble de solides.}
\newcommand{\nbcomp}{1}
\newcommand{\systemes}{Maxpid}
\newcommand{\systemesnum}{26}
\newcommand{\systemessansaccent}{Maxpid}
\newcommand{\ilot}{2}
\newcommand{\ilotstr}{02}
\newcommand{\dossierilot}{\detokenize{Ilot_02 Maxpid}}
\newcommand{\imageun}{Maxpid}

\newcommand{\scilabxcos}{\href{https://raw.githubusercontent.com/Costadoat/Sciences-Ingenieur/master/Systemes/Maxpid/Maxpid_complet.zcos}{Modèle complet du  Maxpid}}
\newcommand{\matlabsimscape}{\href{https://raw.githubusercontent.com/Costadoat/Sciences-Ingenieur/master/Systemes/Maxpid/Maxpid_Simscape.zip}{Modèle Simscape}}
\newcommand{\solidworks}{\href{https://raw.githubusercontent.com/Costadoat/Sciences-Ingenieur/master/Systemes/Maxpid/Maxpid_Solidworks.zip}{Modèle Solidworks}}
\newcommand{\miseenoeuvre}{\href{https://raw.githubusercontent.com/Costadoat/Sciences-Ingenieur/master/Systemes/Maxpid/Maxpid_MO/Maxpid_MO.pdf}{Mise en oeuvre}}
\newcommand{\edrawings}{\href{https://raw.githubusercontent.com/Costadoat/Sciences-Ingenieur/master/Systemes/Maxpid/Maxpid.EASM}{Fichier eDrawing}}
\newcommand{\experimental}{\href{https://raw.githubusercontent.com/Costadoat/Sciences-Ingenieur/master/Systemes/Maxpid/Maxpid_experimental.zip}{Analyse de résultats expérimentaux}}
\newcommand{\schemacinematique}{Maxpid_cinematique}
