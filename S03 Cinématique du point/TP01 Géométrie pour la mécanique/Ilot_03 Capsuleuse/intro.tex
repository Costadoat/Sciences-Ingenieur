\newcommand{\id}{148}
\newcommand{\nom}{Géométrie pour la mécanique}
\newcommand{\sequence}{03}
\newcommand{\nomsequence}{Cinématique du point}
\newcommand{\num}{01}
\newcommand{\type}{TP}
\newcommand{\descrip}{Déterminer une fermeture géométrique et vérifier expérimentalement.}
\newcommand{\competences}{B2-14: Modéliser la cinématique d'un ensemble de solides.}
\newcommand{\nbcomp}{1}
\newcommand{\systemes}{Capsuleuse}
\newcommand{\systemesnum}{50}
\newcommand{\systemessansaccent}{Capsuleuse}
\newcommand{\ilot}{3}
\newcommand{\ilotstr}{03}
\newcommand{\dossierilot}{\detokenize{Ilot_03 Capsuleuse}}
\newcommand{\imageun}{Capsuleuse}

\newcommand{\matlabsimscape}{\href{https://github.com/Costadoat/Sciences-Ingenieur/raw/master/Systemes/Capsuleuse/Capsuleuse_Simscape.zip}{Modèle Simscape}}
\newcommand{\solidworks}{\href{https://github.com/Costadoat/Sciences-Ingenieur/raw/master/Systemes/Capsuleuse/Capsuleuse_Solidworks.zip}{Modèles Solidworks}}
\newcommand{\miseenoeuvre}{\href{https://github.com/Costadoat/Sciences-Ingenieur/raw/master/Systemes/Capsuleuse/Capsuleuse_MO/Capsuleuse_MO.pdf}{Mise en oeuvre de la capsuleuse}}
\newcommand{\edrawings}{\href{https://github.com/Costadoat/Sciences-Ingenieur/raw/master/Systemes/Capsuleuse/Capsuleuse.EASM}{Modèle eDrawings}}
\newcommand{\experimental}{\href{https://github.com/Costadoat/Sciences-Ingenieur/raw/master/Systemes/Capsuleuse/Capsuleuse_experimental.zip}{Analyse de résultats expérimentaux}}
\newcommand{\schemacinematique}{Capsuleuse_cinematique}
