\newcommand{\id}{148}
\newcommand{\nom}{Géométrie pour la mécanique}
\newcommand{\sequence}{03}
\newcommand{\nomsequence}{Cinématique du point}
\newcommand{\num}{01}
\newcommand{\type}{TP}
\newcommand{\descrip}{Déterminer une fermeture géométrique et vérifier expérimentalement.}
\newcommand{\competences}{B2-14: Modéliser la cinématique d'un ensemble de solides.}
\newcommand{\nbcomp}{1}
\newcommand{\systemes}{Plateforme Stewart}
\newcommand{\systemesnum}{57}
\newcommand{\systemessansaccent}{Plateforme Stewart}
\newcommand{\ilot}{4}
\newcommand{\ilotstr}{04}
\newcommand{\dossierilot}{\detokenize{Ilot_04 Plateforme Stewart}}
\newcommand{\imageun}{Plateforme}

\newcommand{\matlabsimscape}{\href{https://github.com/Costadoat/Sciences-Ingenieur/raw/master/Systemes/Plateforme Stewart/Plateforme_Stewart_Simscape.zip}{Modèle Simscape}}
\newcommand{\solidworks}{\href{https://github.com/Costadoat/Sciences-Ingenieur/raw/master/Systemes/Plateforme Stewart/Plateforme_Stewart_Solidworks.zip}{Modèle Solidworks}}
\newcommand{\edrawings}{\href{https://github.com/Costadoat/Sciences-Ingenieur/raw/master/Systemes/Plateforme Stewart/Plateforme_Stewart.EASM}{Modèle eDrawings}}
\newcommand{\test}{Stewart_param1}
\newcommand{\testi}{Stewart_param2}
\newcommand{\testii}{Stewart_param3}
\newcommand{\testiii}{Stewart_param4}
\newcommand{\testiiii}{Stewart_euler}
