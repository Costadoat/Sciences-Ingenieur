\input{../../../headers/tpheaders.tex}

\prob{Modéliser la loi d'entrée/sortie géométrique d'un système} \\

\graphicspath{{../../../img/}}
\begin{center}
\def\svgwidth{\columnwidth}
\input{"../../../img/triptyque.pdf_tex"}
\end{center}

La démarche de l’ingénieur permet :
\begin{itemize}
 \item De vérifier les performances attendues d’un système, par évaluation de l’écart entre un cahier des charges et les réponses expérimentales (écart 1),
 \item De proposer et de valider des modèles d’un système à partir d’essais, par évaluation de l’écart entre les performances mesurées et les performances simulées (écart 2),
 \item De prévoir le comportement à partir de modélisations, par l’évaluation de l’écart entre les performances simulées et les performances attendues du cahier des charges (écart 3).
\end{itemize}

\documentsressource{Pour ce TP, vous aurez à votre diposition les documents suivants:
\begin{itemize}
 \item La \miseenoeuvre\ du système,
 \item de la procédure d'utilisation de Simscape disponible à la page \pageref{proceduresimscape},
 \item Les divers documents des \urlsysteme.
\end{itemize}}

\newpage

\section{Modélisation géométrique} 

Des données sur le système sont disponibles ici: \urlsysteme.

\question{Écrire les vecteurs $\overrightarrow{O_FB_i}$, $\overrightarrow{B_iA_i}$ et $\overrightarrow{A_iO_M}$ dans les bases respectives $B_F(\overrightarrow{x_F},\overrightarrow{y_F},\overrightarrow{z_F})$, $B_i(\overrightarrow{x_i},\overrightarrow{y_i},\overrightarrow{z_i})$ et $B_M(\overrightarrow{x_M},\overrightarrow{y_M},\overrightarrow{z_M})$. On mesurera $\left|\overrightarrow{O_FB_i}\right|$ et $\left|\overrightarrow{O_MA_i}\right|$ directement sur le système et on prendra $\left|\overrightarrow{A_iB_i}\right|=L_i(t)$ variable. On prendra aussi pour simplifier $\overrightarrow{O_FO_M}=z(t).\overrightarrow{z_F}$ et $(\overrightarrow{x_F},\overrightarrow{x_M})=\frac{\pi}{6}$.}

\question{Donner la relation qui existe entre ces vecteurs.}

\question{Déterminer la norme de la longueur $L_i(t)$ de chaque vérin en fonction des paramètres du système.}

\section{Vérification par la simulation}

\question{Simuler le modèle Simulink sans le modifier, vérifier les données affichées.}

\question{Éditer le modèle en recopiant la formule de la première partie dans le bloc fonction et comparer les résultats des deux modèles.}

\section{Vérification expérimentale}

\question{Filmer le mouvement de la plateforme Stewart dans la même vue que celle du schéma cinématique.}

\question{A l'aide du logiciel Tracking repérer les trajectoires des points du schéma cinématique et valider les résultats précédents.}

\newpage

\proceduresimscape

\ifdef{\public}{\end{document}}

\newpage

\pagestyle{correction}

\section{Correction}

\cor{$\overrightarrow{AB}=a\cdot\overrightarrow{y_0}$, $\overrightarrow{AC}=l(t)\cdot\overrightarrow{x_1}$ et $\overrightarrow{BC}=b\cdot\overrightarrow{x_2}$, avec a=112mm et b=81mm.}

\cor{$\overrightarrow{AC}=\overrightarrow{AB}+\overrightarrow{BC}$.}

\cor{\begin{eqnarray}
l(t)\cdot cos\theta_1=b\cdot cos\theta_2\\ 
l(t)\cdot sin\theta_1=a+b\cdot sin\theta_2
\end{eqnarray}}

\cor{\begin{eqnarray}
tan\theta_1=\frac{a+b\cdot sin\theta_2}{b\cdot cos\theta_2}
\end{eqnarray}}

\cor{\begin{eqnarray}
\theta_1=arctan\left(\frac{a+b\cdot sin\theta_2}{b\cdot cos\theta_2}\right)
\end{eqnarray}}

\cor{\begin{eqnarray}
b\cdot sin\theta_1\cdot cos\theta_2=a\cdot cos\theta_1+b\cdot sin\theta_2\cdot cos\theta_1 \nonumber \\
b\cdot \left(sin\theta_1\cdot cos\theta_2-sin\theta_2\cdot cos\theta_1\right)=a\cdot cos\theta_1 \nonumber \\
b\cdot sin\left(\theta_1-\theta_2\right)=a\cdot cos\theta_1 \nonumber \\
\theta_1-\theta_2=arcsin\left(\frac{a}{b}\cdot cos\theta_1\right) \nonumber \\
\theta_2=\theta_1-arcsin\left(\frac{a}{b}\cdot cos\theta_1\right)
\end{eqnarray}}


\end{document}
