\newcommand{\id}{148}
\newcommand{\nom}{Géométrie pour la mécanique}
\newcommand{\sequence}{03}
\newcommand{\nomsequence}{Cinématique du point}
\newcommand{\num}{01}
\newcommand{\type}{TP}
\newcommand{\descrip}{Déterminer une fermeture géométrique et vérifier expérimentalement.}
\newcommand{\competences}{B2-14: Modéliser la cinématique d'un ensemble de solides.}
\newcommand{\nbcomp}{1}
\newcommand{\systemes}{Moby Crea}
\newcommand{\systemesnum}{55}
\newcommand{\systemessansaccent}{Moby Crea}
\newcommand{\ilot}{5}
\newcommand{\ilotstr}{05}
\newcommand{\dossierilot}{\detokenize{Ilot_05 Moby Crea}}
\newcommand{\imageun}{Moby_Crea}

\newcommand{\matlabsimscape}{\href{https://raw.githubusercontent.com/Costadoat/Sciences-Ingenieur/master/Systemes/Moby Crea/Mobycrea_Simscape.zip}{Modèle Simscape}}
\newcommand{\experimental}{\href{https://raw.githubusercontent.com/Costadoat/Sciences-Ingenieur/master/Systemes/Moby Crea/MobyCrea_experimental.zip}{Analyse de résultats expérimentaux}}
\newcommand{\miseenoeuvre}{\href{https://raw.githubusercontent.com/Costadoat/Sciences-Ingenieur/master/Systemes/Moby Crea/MobyCrea_MO/MobyCrea_MO.pdf}{Mise en oeuvre}}
\newcommand{\scilabxcos}{\href{https://raw.githubusercontent.com/Costadoat/Sciences-Ingenieur/master/Systemes/Moby Crea/MobyCrea.zcos}{Modèle Scilab}}
\newcommand{\schemacinematique}{MobyCrea_cinematique}
