% Paquets généraux
\documentclass[a4paper,12pt,titlepage]{article}
\usepackage[T1]{fontenc}
\usepackage[utf8]{inputenc}
\usepackage[french]{babel}
\usepackage[gen]{eurosym}
%\usepackage[dvips]{graphicx}
\usepackage{fancyhdr}
\usepackage{pdfpages} 
\usepackage{multido}
\usepackage{hyperref}
%\usepackage{textcomp}
%\usepackage{aeguill}
\usepackage{schemabloc}
\usepackage[bitstream-charter]{mathdesign}

\newcommand{\auteurun}{Renaud Costadoat}
\newcommand{\auteurdeux}{Françoise Puig}
\newcommand{\institute}{Lycée Dorian}


\usepackage{color}
\usepackage{xcolor}
\usepackage{colortbl}
\usepackage{helvet}
\renewcommand{\familydefault}{\sfdefault}
\usepackage{amsfonts}
\usepackage{amsmath}
%\usepackage{xspace}
\usepackage{varioref}
\usepackage{tabularx}
%\usepackage{floatflt}
\usepackage{graphics}
\usepackage{wrapfig}
\usepackage{textcomp}
\usepackage{tikz}
\usepackage{wrapfig}
\usepackage{gensymb}
\usepackage[european]{circuitikz}
\usetikzlibrary{babel}
\usepackage{ifthen}
\usepackage{cancel}
\usepackage{etoolbox}
\usepackage{multirow}
%\usepackage{boxedminipage}
\definecolor{gris25}{gray}{0.75}
\definecolor{bleu}{RGB}{18,33,98}
\definecolor{bleuf}{RGB}{42,94,171}
\definecolor{bleuc}{RGB}{231,239,247}
\definecolor{rougef}{RGB}{185,18,27}
\definecolor{rougec}{RGB}{255,188,204}%255,230,231
\definecolor{vertf}{RGB}{103,126,82}
\definecolor{vertc}{RGB}{220,255,191}
\definecolor{forestgreen}{rgb}{0.13,0.54,0.13}
\definecolor{blcr}{rgb}{0.59,0.69,0.84}
\definecolor{blfr}{rgb}{0.32,0.51,0.75}
\definecolor{orfr}{rgb}{0.90,0.42,0.15}
\definecolor{orcr}{rgb}{0.90,0.65,0.50}
\definecolor{orangef}{rgb}{0.659,0.269,0.072}
\definecolor{orange}{rgb}{0.58,0.35,0.063}
\definecolor{orangec}{rgb}{0.43,0.32,0.25}
\definecolor{rcorrect}{rgb}{0.6,0,0}
\definecolor{sequence}{rgb}{0.75,0.75,0.75}
\definecolor{competences}{rgb}{0.61,0.73,0.35}
\definecolor{grisf}{HTML}{222222}
\definecolor{grisc}{HTML}{636363}
\definecolor{normal}{HTML}{4087c4}
\definecolor{info}{HTML}{5bc0de}
\definecolor{success}{RGB}{92,184,92}
\definecolor{warning}{RGB}{240,173,78}
\definecolor{danger}{RGB}{217,83,79}
\hypersetup{                    % parametrage des hyperliens
    colorlinks=true,                % colorise les liens
    breaklinks=true,                % permet les retours à la ligne pour les liens trop longs
    urlcolor= blfr,                 % couleur des hyperliens
    linkcolor= orange,                % couleur des liens internes aux documents (index, figures, tableaux, equations,...)
    citecolor= forestgreen                % couleur des liens vers les references bibliographiques
    }

% Mise en page
\pagestyle{fancy}

\setlength{\hoffset}{-18pt}

\setlength{\oddsidemargin}{0pt} 	% Marge gauche sur pages impaires
\setlength{\evensidemargin}{0pt} 	% Marge gauche sur pages paires
\setlength{\marginparwidth}{00pt} 	% Largeur de note dans la marge
\setlength{\headwidth}{481pt} 	 	% Largeur de la zone de tête (17cm)
\setlength{\textwidth}{481pt} 	 	% Largeur de la zone de texte (17cm)
\setlength{\voffset}{-18pt} 		% Bon pour DOS
\setlength{\marginparsep}{7pt}	 	% Séparation de la marge
\setlength{\topmargin}{-30pt} 		% Pas de marge en haut
\setlength{\headheight}{35pt} 		% Haut de page
\setlength{\headsep}{20pt} 		% Entre le haut de page et le texte
\setlength{\footskip}{30pt} 		% Bas de page + séparation
\setlength{\textheight}{700pt} 		% Hauteur de l'icone zone de texte (25cm)
\setlength\fboxrule{1 pt}
\renewcommand{\baselinestretch}{1}
\setcounter{tocdepth}{1}
\newcommand{\cadre}[2]
{\fbox{
  \begin{minipage}{#1\linewidth}
   \begin{center}
    #2\\
   \end{center}
  \end{minipage}
 }
}

\newcounter{num_quest} \setcounter{num_quest}{0}
\newcounter{num_rep} \setcounter{num_rep}{0}
\newcounter{num_cor} \setcounter{num_cor}{0}

\newcommand{\question}[1]{\refstepcounter{num_quest}\par
~\ \\ \parbox[t][][t]{0.15\linewidth}{\textbf{Question \arabic{num_quest}}}\parbox[t][][t]{0.93\linewidth}{#1}\par
}


\newcommand{\reponse}[1]
{\refstepcounter{num_rep}
\noindent
\rule{\linewidth}{.5pt}
\textbf{Question \arabic{num_rep}:}
\multido{\i=1+1}{#1}{~\ \\}
}

\newcommand{\cor}
{\refstepcounter{num_cor}
\noindent
\rule{\linewidth}{.5pt}
\textbf{Question \arabic{num_cor}:} \\
}

\newcommand{\titre}[1]
{\begin{center}
\cadre{0.8}{\huge #1} 
\end{center}
}


% En tête et pied de page
\fancypagestyle{normal}{%
  \fancyhf{}
\lhead{Mise en \oe uvre de \nomtitre}
\rhead{\includegraphics[width=2cm]{../../../img/logo}\hspace{2pt}}
\ifdef{\auteurdeux}{\lfoot{\auteurun,\auteurdeux}}{\lfoot{\auteurun}}
\cfoot{Page \thepage}}

\fancypagestyle{correction}{%
  \fancyhf{}
  \lhead{\colorbox{danger}{\begin{minipage}{0.65\paperwidth} \textcolor{white}{\textbf{Correction}} \end{minipage}} }
  \rhead{\includegraphics[width=2cm]{../../../img/logo}}
  \ifdef{\auteurdeux}{\lfoot{\auteurun,\auteurdeux}}{\lfoot{\auteurun}}
  \rfoot{\colorbox{danger}{\begin{minipage}{0.5\paperwidth} \begin{flushright}\textcolor{white}{\textbf{Correction}}\end{flushright} \end{minipage}} }}

\renewcommand{\footrulewidth}{0.4pt}

\usepackage{eso-pic}
\newcommand{\BackgroundPic}{%
\put(0,0){%
\parbox[b][\paperheight]{\paperwidth}{%
\vfill
\begin{center}
\hspace{0.5cm}\vspace{0.5cm}
\includegraphics[width=\paperwidth,height=\paperheight,%
keepaspectratio]{../../../img/fond3}%
\end{center}
\vfill
}}}

\newcommand{\BackgroundPicdeux}{%
\put(25,-30){%
\parbox[b][\paperheight]{\paperwidth}{%
\vfill
\begin{center}
\includegraphics[width=\paperwidth,height=\paperheight,%
keepaspectratio]{../../../img/fond4}%
\end{center}
\vfill
}}}

\newcounter{rowcounter}
\stepcounter{rowcounter}
\newcolumntype{E}{>{\bfseries \arabic{rowcounter}. \stepcounter{rowcounter}}l}

\begin{document}

\pagestyle{empty}

\vspace*{-3\baselineskip}

\AddToShipoutPicture*{\BackgroundPic}

\begin{tabular}{>{\columncolor{gray!00}}m{.3\linewidth} m{.3\linewidth} >{\columncolor{gray!00}}m{.3\linewidth}}
Système : \nomsysteme &  \multirow{3}{*}{\hspace{1cm}\includegraphics[height=1.5cm]{../../../img/logo}} &  \begin{flushright} \multirow{4}{*}{\hspace{1cm}}\end{flushright}\\
Document : Mise en \oe uvre\\
 \institute \\
 \auteurun\\
 \auteurdeux
\end{tabular}

\vspace{1cm}

\begin{center}\huge{Mise en \oe uvre \nomtitre}\end{center}

\vspace{2cm}

\begin{center}\includegraphics[height=5cm]{/home/renaud/Documents/Renaud/GitHub/django_education/systemes/\image}\end{center}

\newpage

\AddToShipoutPicture{\BackgroundPicdeux}

\pagestyle{normal}


\begin{tabular}{|Ep{0.4\linewidth}|p{.5\linewidth}|}
\hline
& Mettre le système sous tension,
&\raisebox{-0.9\totalheight}{\def\svgwidth{0.9\linewidth}
 \input{img/page_1.pdf_tex}}\tabularnewline\hline
& Déverrouiller le bouton d'arrêt d'urgence,
Appuyer sur Armer le système et sur Réinitialiser,
&\raisebox{-0.9\totalheight}{\def\svgwidth{0.9\linewidth}
 \input{img/page_2.pdf_tex}}\tabularnewline\hline
 & Cliquer sur l'icône du logiciel de pilotage, 
&\raisebox{-0.9\totalheight}{\includegraphics[width=0.5\linewidth]{img/controlX_00}}
\tabularnewline\hline
 & Un modèle didactique du système permet de visualiser l'état des capteurs,  
&\raisebox{-0.9\totalheight}{\def\svgwidth{0.9\linewidth}
 \input{img/page_3.pdf_tex}}\tabularnewline\hline
 & En cliquant sur BO/BF sur le Menu de configuration, il est possible de sélectionner le mode de pilotage BO/BF su système.
 &\raisebox{-0.9\totalheight}{\includegraphics[width=0.5\linewidth]{img/controlX_02}}
\tabularnewline\hline
\end{tabular}

\section{Réponse temporelle}

\setcounter{rowcounter}{1}

\begin{tabular}{|Ep{0.4\linewidth}|p{.5\linewidth}|}
\hline
 & Cliquer sur Analyse temporelle sur le menu de configuration et sur les Onglets d'affichage,\newline Définir l'entrée du mouvement,
 \begin{enumerate}
  \item Sélectionner la forme de l'entrée,
  \item Régler l'entrée (si BO en V, si BF en mm),
  \item Vérifier sur la prévisualisation que la forme correspond,
  \item Générer l'entrée
  \end{enumerate}
&\raisebox{-0.9\totalheight}{\def\svgwidth{0.9\linewidth}
 \input{img/page_4.pdf_tex}}\tabularnewline\hline
& Visualiser la sortie ,
 \begin{enumerate}
  \item Le résultat est affiché dans la partie Réponses temporelles,
  \item Il est possible d'identifier le résultat en choisissant le Type d'identification et en jouant sur son paramétrage..
  \end{enumerate}
&\raisebox{-0.9\totalheight}{\def\svgwidth{0.9\linewidth}
\input{img/page_6.pdf_tex}}\tabularnewline\hline
\end{tabular}

\section{Étude sollicitations harmoniques}

\setcounter{rowcounter}{1}

\begin{tabular}{|Ep{0.4\linewidth}|p{.5\linewidth}|}
\hline
 & Cliquer sur Analyse harmonique sur le menu de configuration et sur les Onglets d'affichage,
 \begin{enumerate}
  \item Régler les paramètres de l'analyse
  \item Lancer l'analyse,
  \item Les diagrammes de Bode s'affichent sur l'interface directement.
  \end{enumerate}
&\raisebox{-0.9\totalheight}{\def\svgwidth{0.9\linewidth}
 \input{img/page_5.pdf_tex}}\tabularnewline\hline
\end{tabular}

\section{Réglage des correcteurs}

\setcounter{rowcounter}{1}

\begin{tabular}{|Ep{0.4\linewidth}|p{.5\linewidth}|}
\hline
 & En cliquant sur Correcteur sur le Menu de configuration, il est possible de régler la correction du système
 &\raisebox{-0.9\totalheight}{\includegraphics[width=0.8\linewidth]{img/controlX_06}}
\tabularnewline\hline
\end{tabular}
\end{document}
