\newcommand{\num}{01}
% Paquets généraux
\documentclass[a4paper,12pt,titlepage,twoside]{article}
\usepackage[T1]{fontenc}
\usepackage[utf8]{inputenc}
\usepackage[french]{babel}
\usepackage[gen]{eurosym}
%\usepackage[dvips]{graphicx}
\usepackage{fancyhdr}
\usepackage{pdfpages} 
\usepackage{multido}
\usepackage{enumitem}
\usepackage{hyperref}
%\usepackage{textcomp}
%\usepackage{aeguill}
\usepackage{schemabloc}
\usepackage[bitstream-charter]{mathdesign}

\newcommand{\id}{30}
\newcommand{\nom}{Calculs d'hyperstatisme}
\newcommand{\sequence}{04}
\newcommand{\nomsequence}{Liaisons entre les solides}
\newcommand{\num}{03}
\newcommand{\type}{TD}
\newcommand{\descrip}{En appliquant les règles de la théorie des mécanisme, déterminer le degré d'hyperstatisme de plusieurs systèmes et proposer des solutions afin de diminuer ce degré}
\newcommand{\competences}{B2-12: Proposer une modélisation des liaisons avec leurs caractéristiques géométriques. \\ &  B2-13: Proposer un modèle cinématique paramétré à partir d'un système réel, d'une maquette numérique ou d'u \\ &  B2-17: Simplifier un modèle de mécanisme. \\ &  B2-18: Modifier un modèle pour le rendre isostatique.}
\newcommand{\nbcomp}{4}
\newcommand{\systemes}{E.P.A.S, Machine d'essai de traction}
\newcommand{\systemesnum}{14, 13}
\newcommand{\systemessansaccent}{E.P.A.S, Machine d'essai de traction}
\newcommand{\ilot}{3}
\newcommand{\ilotstr}{03}
\newcommand{\dossierilot}{\detokenize{Ilot_03 E.P.A.S, Machine d'essai de traction}}
\newcommand{\imageun}{EPAS}
\newcommand{\imagedeux}{Machine_dessai_de_traction}


\newcommand{\institute}{Lycée Dorian}


\usepackage{color}
\usepackage{xcolor}
\usepackage{colortbl}
\usepackage{helvet}
\renewcommand{\familydefault}{\sfdefault}
\usepackage[frenchmath]{newtxsf} % for sans serif symbols
%\usepackage{amsfonts}
%\usepackage{amsmath}
%\usepackage{xspace}
\usepackage{varioref}
\usepackage{tabularx}
%\usepackage{floatflt}
\usepackage{graphics}
\usepackage{wrapfig}
\usepackage{textcomp}
\usepackage{tikz}
\usepackage{wrapfig}
\usepackage{gensymb}
\usepackage[european]{circuitikz}
\usetikzlibrary{babel}
\usepackage{ifthen}
\usepackage{cancel}
\usepackage{etoolbox}
\usepackage{multirow}
%\usepackage{boxedminipage}
\definecolor{gris25}{gray}{0.75}
\definecolor{bleu}{RGB}{18,33,98}
\definecolor{bleuf}{RGB}{42,94,171}
\definecolor{bleuc}{RGB}{231,239,247}
\definecolor{rougef}{RGB}{185,18,27}
\definecolor{rougec}{RGB}{255,188,204}%255,230,231
\definecolor{vertf}{RGB}{103,126,82}
\definecolor{vertc}{RGB}{220,255,191}
\definecolor{forestgreen}{rgb}{0.13,0.54,0.13}
\definecolor{blcr}{rgb}{0.59,0.69,0.84}
\definecolor{blfr}{rgb}{0.32,0.51,0.75}
\definecolor{orfr}{rgb}{0.90,0.42,0.15}
\definecolor{orcr}{rgb}{0.90,0.65,0.50}
\definecolor{orangef}{rgb}{0.659,0.269,0.072}
\definecolor{orange}{rgb}{0.58,0.35,0.063}
\definecolor{orangec}{rgb}{0.43,0.32,0.25}
\definecolor{rcorrect}{rgb}{0.6,0,0}
\definecolor{sequence}{rgb}{0.75,0.75,0.75}
\definecolor{competences}{rgb}{0.61,0.73,0.35}
\definecolor{grisf}{HTML}{222222}
\definecolor{grisc}{HTML}{636363}
\definecolor{normal}{HTML}{4087c4}
\definecolor{info}{HTML}{5bc0de}
\definecolor{success}{RGB}{92,184,92}
\definecolor{warning}{RGB}{240,173,78}
\definecolor{danger}{RGB}{217,83,79}
\hypersetup{                    % parametrage des hyperliens
    colorlinks=true,                % colorise les liens
    breaklinks=true,                % permet les retours à la ligne pour les liens trop longs
    urlcolor= blfr,                 % couleur des hyperliens
    linkcolor= orange,                % couleur des liens internes aux documents (index, figures, tableaux, equations,...)
    citecolor= forestgreen                % couleur des liens vers les references bibliographiques
    }

% Mise en page
\pagestyle{fancy}

\setlength{\hoffset}{-18pt}

\setlength{\oddsidemargin}{0pt} 	% Marge gauche sur pages impaire2s
\setlength{\evensidemargin}{0pt} 	% Marge gauche sur pages paires
\setlength{\marginparwidth}{00pt} 	% Largeur de note dans la marge
\setlength{\headwidth}{481pt} 	 	% Largeur de la zone de tête (17cm)
\setlength{\textwidth}{481pt} 	 	% Largeu\textbf{r de la zone de texte (17cm)
\setlength{\voffset}{-18pt} 		% Bon pour DOS
\setlength{\marginparsep}{7pt}	 	% Séparation de la marge
\setlength{\topmargin}{-30pt} 		% Pas de marge en haut
\setlength{\headheight}{35pt} 		% Haut de page
\setlength{\headsep}{20pt} 		% Entre le haut de page et le texte
\setlength{\footskip}{30pt} 		% Bas de\textbf{ page + séparation
\setlength{\textheight}{700pt} 		% Hauteur de l'icone zone de texte (25cm)
\setlength\fboxrule{1 pt}
\renewcommand{\baselinestretch}{1}
\setcounter{tocdepth}{1}
\newcommand{\cadre}[2]
{\fbox{
  \begin{minipage}{#1\linewidth}
   \begin{center}
    #2\\
   \end{center}
  \end{minipage}
 }
}

\newcounter{num_quest} \setcounter{num_quest}{0}
\newcounter{num_rep} \setcounter{num_rep}{0}
\newcounter{num_cor} \setcounter{num_cor}{0}

\newcommand{\question}[1]{\refstepcounter{num_quest}\par
~\ \\ \textbf{Question \arabic{num_quest} : }#1\label{q\the\value{num_quest}}\par
}


\newcommand{\feuilleDR}[1]{
	\begin{tikzpicture}
		\draw[gray!30](0,0)grid[step=0.5cm](\linewidth,#1);
	\end{tikzpicture}
}


\newcommand{\reponse}[4][1]
{\noindent
\parbox{\textwidth}{
\rule{\linewidth}{.5pt}\\
\textbf{Question\ifthenelse{#1>1}{s}{} \multido{}{#1}{%
\refstepcounter{num_rep}\ref{q\the\value{num_rep}} }:} ~\ \\
\ifdef{\public}{#3 \ifthenelse{#2>0}{~\ \\ 	\feuilleDR{#2}}}{#4}
}}

\newcommand{\cor}
{\refstepcounter{num_cor}
\noindent
\rule{\linewidth}{.5pt}
\textbf{Question \arabic{num_cor}:} \\
}

\newcommand{\titre}[1]
{\begin{center}
\cadre{0.8}{\huge #1} 
\end{center}
}

\newcommand{\finsujet}
{
    \begin{center}
    \Large{FIN}
    \end{center}

    \cleardoublepage

	\def\public

    \ifdef{\public}{\pagestyle{docreponse}}{\pagestyle{correction}}

    \ifdef{\public}{
        \begin{tikzpicture} 
            \draw (0,0) rectangle (2,2);
            \draw (0,0) -- (2,2);
            \draw (1.5,0.5) node {\large 20};
            \draw (2.5,0) rectangle (16,2);
            \draw (4.5,1.7) node {\large Commentaires:};
        \end{tikzpicture}
        \ifdefined\competences
	      ~\ \\ ~\ \\
	      \centering
		  \foreach \competence in \competences
		  {
			\fbox{\begin{tikzpicture} 
			\draw (0,0) circle[radius=0.5];
		    \node at (0.5,0) [anchor=west] {\competence};
   	     \end{tikzpicture}
   	    	}
   	    }
	   	\fi
	   	 ~\ \\
	}

    ~\ \\
}

% En tête et pied de page
\fancypagestyle{normal}{%
\fancyhf{}
\lhead{\sujet}
\rhead{\includegraphics[width=2cm]{../../../img/logo}\hspace{2pt}}
\ifdef{\auteurdeux}{\lfoot{\auteurun,\ \auteurdeux}}{\lfoot{\auteurun}}
\rfoot{\nom}
}

\fancypagestyle{docreponse}{%
\fancyhf{}
\fancyhead[LO]{NOM Prénom: .............................}
\rhead{\includegraphics[width=2cm]{../../../img/logo}\hspace{2pt}}
\ifdef{\auteurdeux}{\lfoot{\auteurun,\ \auteurdeux}}{\lfoot{\auteurun}}
\rfoot{\nom}
}

\fancypagestyle{correction}{%
  \fancyhf{}
  \lhead{\colorbox{danger}{\begin{minipage}{0.65\paperwidth} \textcolor{white}{\textbf{Correction}} \end{minipage}} }
  \rhead{\includegraphics[width=2cm]{../../../img/logo}}
  \ifdef{\auteurdeux}{\lfoot{\auteurun,\auteurdeux}}{\lfoot{\auteurun}}
  \rfoot{\colorbox{danger}{\begin{minipage}{0.5\paperwidth} \begin{flushright}\textcolor{white}{\textbf{Correction}}\end{flushright} \end{minipage}} }}

\renewcommand{\footrulewidth}{0.4pt}

\usepackage{eso-pic}
\newcommand{\BackgroundPic}{%
\put(0,0){%
\parbox[b][\paperheight]{\paperwidth}{%
\vfill
\begin{center}
\hspace{0.5cm}\vspace{0.5cm}
\includegraphics[width=\paperwidth,height=\paperheight,%
keepaspectratio]{../../../img/fond3}%
\end{center}
\vfill
}}}

\newcommand{\BackgroundPicdeux}{%
\put(25,-30){%
\parbox[b][\paperheight]{\paperwidth}{%
\vfill
\begin{center}
\includegraphics[width=\paperwidth,height=\paperheight,%
keepaspectratio]{../../../img/fond4}%
\end{center}
\vfill
}}}

\begin{document}

\AddToShipoutPicture{\BackgroundPicdeux}

\pagestyle{normal}


\section{Le système et le produit}

\question{Comment s'appelle le document par lequel le demandeur exprime son besoin au travers d’exigences et qui sert de contrat entre le client et le concepteur ?}

\question{D'après le diagramme de la figure \ref{figure1}, quels sont les acteurs du système ?}

\question{De manière générale, parmi les diagrammes suivants, sur lesquels est-il possible de trouver un acteur : diagramme des cas d'utilisation, diagramme de séquence, diagramme d'exigences et diagramme de contexte.}

\question{Comment s'appelle un élément qui n'est pas un acteur mais ne fait pas partie du système, même s'il interagit avec lui ?}

\question{Quel élément permet de stocker et de restituer l'énergie ? A quel sous-système transmet-il cette énergie ?}

\question{Quel élément permet de moduler l'énergie ?}

\question{Quel élément permet de connaître la position courante de la lisse ?}

\question{Dans la configuration la plus rapide, combien de temps dure la montée de la lisse ? Quelle est alors la vitesse moyenne de la montée de la barrière en $rad.s^{-1}$}

\question{Sur quelle.s figure.s peut-on voir un diagramme de contexte ?}

\question{Comment appelle-t-on les éléments réels qui permettent de répondre au réel ? Par quel élément SysMl sont-ils représentés ?}

\question{Comment appelle-t-on la relation qui existe lorsqu’un bloc englobe ses parties, le côté du tout indiqué par un losange plein ?}

\question{Comment s'appelle cette relation si les parties ne sont pas nécessaires au tout (indiquée par un losange blanc).}

\question{Compléter la phrase suivante: Chaque élément du système peut être décomposé en sous-composants qui apparaissent dans un diagramme de blocs interne, des ..... circulent entre ces composants et interagissent avec les composants par l’intermédiaire de ......}

\newpage

\section{Diagrammes}

\begin{figure}[ht!]
\begin{center}
 \includegraphics[width=0.5\linewidth]{img/diagram1}
\end{center}
\caption{\label{figure1} Diagramme 1}
\end{figure}

\begin{figure}[ht!]
\begin{center}
 \includegraphics[width=\linewidth]{img/diagram2}
\end{center}
\caption{\label{figure2} Diagramme 2}
\end{figure}

\newpage

\begin{figure}[ht!]
\begin{center}
 \includegraphics[width=0.7\linewidth]{img/diagram3}
\end{center}
\caption{\label{figure3} Diagramme 3}
\end{figure}

\begin{figure}[ht!]
\begin{center}
 \includegraphics[width=\linewidth]{img/diagram4}
\end{center}
\caption{\label{figure4} Diagramme 4}
\end{figure}

\newpage

\begin{figure}[ht!]
\begin{center}
 \includegraphics[width=\linewidth]{img/diagram5}
\end{center}
\caption{\label{figure5} Diagramme 5}
\end{figure}

\begin{figure}[ht!]
\begin{center}
 \includegraphics[width=\linewidth]{img/diagram6}
\end{center}
\caption{\label{figure6} Diagramme 6}
\end{figure}

\finsujet

\reponse{2}{}{Le cahier des charges}

\reponse{2}{}{L'usager et le technicien.}

\reponse{2}{}{Sur le diagramme des cas d'utilisation, diagramme de séquence et le diagramme de contexte.}

\reponse{2}{}{Un élément du milieu extérieur.}

\reponse{2}{}{Le ressort.}

\reponse{2}{}{Le variateur.}

\reponse{2}{}{Le capteur rotatif de position.}

\reponse{2}{}{La remontée dure 1s. Pour parcourir $\dfrac{\pi}{2}rad$, donc $\dfrac{\pi}{2}rad\cdot s^{-1}$.}

\reponse{2}{}{Figure \ref{figure2} et \ref{figure3}.}

\reponse{2}{}{Les solutions techniques.}

\reponse{1}{}{Relation de composition.}

\reponse{1}{}{Relation d'agrégation.}

\reponse{0}{Compléter la phrase suivante: Chaque élément du système peut être décomposé en sous-composants qui apparaissent dans un diagramme de blocs interne, des .......... circulent entre ces composants et interagissent avec les composants par l’intermédiaire de ...........}{Chaque élément du système peut être décomposé en sous-composants qui apparaissent dans un diagramme de blocs interne, des flux circulent entre ces composants et interagissent avec les composants par l’intermédiaire de ports.}

\end{document}
