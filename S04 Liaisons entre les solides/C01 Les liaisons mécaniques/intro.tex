\newcommand{\id}{5}
\newcommand{\nom}{Les liaisons mécaniques}
\newcommand{\sequence}{04}
\newcommand{\nomsequence}{Liaisons entre les solides}
\newcommand{\num}{01}
\newcommand{\type}{C}
\newcommand{\descrip}{Modèles des liaisons, Introduction aux torseurs, Le schéma cinématique}
\newcommand{\competences}{B2-12: Proposer une modélisation des liaisons avec leurs caractéristiques géométriques. \\ &  B2-13: Proposer un modèle cinématique paramétré à partir d'un système réel, d'une maquette numérique ou d'u \\ &  B2-14: Modéliser la cinématique d'un ensemble de solides. \\ &  B2-16: Modéliser une action mécanique.}
\newcommand{\nbcomp}{4}
\newcommand{\systemes}{}
\newcommand{\systemesnum}{}
\newcommand{\systemessansaccent}{}
\newcommand{\ilot}{1}
\newcommand{\ilotstr}{01}
\newcommand{\dossierilot}{\detokenize{Ilot_01 }}

