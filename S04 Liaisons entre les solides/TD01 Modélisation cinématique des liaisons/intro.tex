\newcommand{\id}{28}
\newcommand{\nom}{Modélisation cinématique des liaisons}
\newcommand{\sequence}{04}
\newcommand{\nomsequence}{Liaisons entre les solides}
\newcommand{\num}{01}
\newcommand{\type}{TD}
\newcommand{\descrip}{Mise en place de torseurs cinématiques et représentation par schéma cinématique}
\newcommand{\competences}{B2-12: Proposer une modélisation des liaisons avec leurs caractéristiques géométriques. \\ &  B2-13: Proposer un modèle cinématique paramétré à partir d'un système réel, d'une maquette numérique ou d'u \\ &  B2-14: Modéliser la cinématique d'un ensemble de solides.}
\newcommand{\nbcomp}{3}
\newcommand{\systemes}{Chaise de dentiste, Moteur à explosion}
\newcommand{\systemesnum}{8, 10}
\newcommand{\systemessansaccent}{Chaise de dentiste, Moteur a explosion}
\newcommand{\ilot}{1}
\newcommand{\ilotstr}{01}
\newcommand{\dossierilot}{\detokenize{Ilot_01 Chaise de dentiste, Moteur à explosion}}
\newcommand{\imageun}{Chaise_de_dentiste}
\newcommand{\imagedeux}{Moteur_a_explosion}

