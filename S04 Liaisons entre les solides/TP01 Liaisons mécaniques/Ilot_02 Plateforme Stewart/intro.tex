\newcommand{\id}{54}
\newcommand{\nom}{Liaisons mécaniques}
\newcommand{\sequence}{04}
\newcommand{\nomsequence}{Liaisons entre les solides}
\newcommand{\num}{01}
\newcommand{\type}{TP}
\newcommand{\descrip}{Modélisation d'un solide. Comportement des liaisons mécaniques. Modéliser les mécanismes du laboratoire par un schéma cinématique, paramétré.}
\newcommand{\competences}{A3-04: Identifier et décrire les liens entre les chaines fonctionnelles. \\ &  B2-10: Déterminer les caractéristiques d'un solide ou d'un ensemble de solides indéformables. \\ &  B2-11: Intégrer ou modifier une pièce dans un assemblage à l'aide d'un modeleur volumique 3D. \\ &  B2-12: Proposer une modélisation des liaisons avec leurs caractéristiques géométriques. \\ &  B2-13: Proposer un modèle cinématique paramétré à partir d'un système réel, d'une maquette numérique ou d'u \\ &  B2-14: Modéliser la cinématique d'un ensemble de solides. \\ &  B2-16: Modéliser une action mécanique. \\ &  C2-07: Déterminer les actions mécaniques en statique. \\ &  C3-01: Mener une simulation numérique.  \\ &  C3-02: Résoudre numériquement une équation ou un système d'équations.  \\ &  E1-05: Lire et décoder un document technique. \\ &  E2-01: Choisir un outil de communication adapté à l'interlocuteur.}
\newcommand{\nbcomp}{12}
\newcommand{\systemes}{Plateforme Stewart}
\newcommand{\systemesnum}{57}
\newcommand{\systemessansaccent}{Plateforme Stewart}
\newcommand{\ilot}{2}
\newcommand{\ilotstr}{02}
\newcommand{\dossierilot}{\detokenize{Ilot_02 Plateforme Stewart}}
\newcommand{\imageun}{Plateforme}

\newcommand{\matlabsimscape}{\href{https://github.com/Costadoat/Sciences-Ingenieur/raw/master/Systemes/Plateforme Stewart/Plateforme_Stewart_Simscape.zip}{Modèle Simscape}}
\newcommand{\solidworks}{\href{https://github.com/Costadoat/Sciences-Ingenieur/raw/master/Systemes/Plateforme Stewart/Plateforme_Stewart_Solidworks.zip}{Modèle Solidworks}}
\newcommand{\edrawings}{\href{https://github.com/Costadoat/Sciences-Ingenieur/raw/master/Systemes/Plateforme Stewart/Plateforme_Stewart.EASM}{Modèle eDrawings}}
\newcommand{\test}{Stewart_param1}
\newcommand{\testi}{Stewart_param2}
\newcommand{\testii}{Stewart_param3}
\newcommand{\testiii}{Stewart_param4}
\newcommand{\testiiii}{Stewart_euler}
