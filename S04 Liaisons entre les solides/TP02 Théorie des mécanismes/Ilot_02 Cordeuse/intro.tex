\newcommand{\id}{55}
\newcommand{\nom}{Théorie des mécanismes}
\newcommand{\sequence}{04}
\newcommand{\nomsequence}{Liaisons entre les solides}
\newcommand{\num}{02}
\newcommand{\type}{TP}
\newcommand{\descrip}{Le système de solides, la théorie des mécanismes. Hyperstatisme et mobilités. Proposer des solutions pour le rendre isostatique et justifier les choix de conception}
\newcommand{\competences}{A3-05: Caractériser un constituant de la chaine de puissance. \\ &  B2-12: Proposer une modélisation des liaisons avec leurs caractéristiques géométriques. \\ &  B2-13: Proposer un modèle cinématique paramétré à partir d'un système réel, d'une maquette numérique ou d'u \\ &  B2-17: Simplifier un modèle de mécanisme. \\ &  B2-18: Modifier un modèle pour le rendre isostatique. \\ &  E2-01: Choisir un outil de communication adapté à l'interlocuteur. \\ &  F2-01: Proposer et hiérarchiser des critères de choix. \\ &  F2-02: Choisir les composants de la chaine d'information. \\ &  F2-03: Choisir les composants de la chaine de puissance. \\ &  F2-04: Modifier la commande pour faire évoluer le comportement du système. }
\newcommand{\nbcomp}{10}
\newcommand{\systemes}{Cordeuse}
\newcommand{\systemesnum}{48}
\newcommand{\systemessansaccent}{Cordeuse}
\newcommand{\ilot}{2}
\newcommand{\ilotstr}{02}
\newcommand{\dossierilot}{\detokenize{Ilot_02 Cordeuse}}
\newcommand{\imageun}{Cordeuse}

\newcommand{\videoavi}{\href{https://github.com/Costadoat/Sciences-Ingenieur/raw/master/Systemes/Cordeuse/Corder_raquette_de_tennis.avi}{Comment corder une raquette de tennis}}
\newcommand{\videoavii}{\href{https://github.com/Costadoat/Sciences-Ingenieur/raw/master/Systemes/Cordeuse/Demonstration_cordeuse.avi}{Démonstration de l'utilisation d'une cordeuse}}
\newcommand{\miseenoeuvre}{\href{https://github.com/Costadoat/Sciences-Ingenieur/raw/master/Systemes/Cordeuse/Cordeuse_MO/Cordeuse_MO.pdf}{Mise en oeuvre}}
