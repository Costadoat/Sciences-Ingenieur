\newcommand{\id}{6}
\newcommand{\nom}{Théorie des mécanismes}
\newcommand{\sequence}{04}
\newcommand{\nomsequence}{Liaisons entre les solides}
\newcommand{\num}{02}
\newcommand{\type}{C}
\newcommand{\descrip}{Liaisons équivalentes, hyperstatisme, liaisons en série et en parallèle, théorie des graphes}
\newcommand{\competences}{B2-12: Proposer une modélisation des liaisons avec leurs caractéristiques géométriques. \\ &  B2-13: Proposer un modèle cinématique paramétré à partir d'un système réel, d'une maquette numérique ou d'u \\ &  B2-17: Simplifier un modèle de mécanisme. \\ &  B2-18: Modifier un modèle pour le rendre isostatique. \\ &  C1-04: Proposer une démarche permettant d'obtenir une loi entrée-sortie géométrique.  \\ &  C2-05: Caractériser le mouvement d'un repère par rapport à un autre repère. \\ &  C2-06: Déterminer les relations entre les grandeurs géométriques ou cinématiques. }
\newcommand{\nbcomp}{7}
\newcommand{\systemes}{}
\newcommand{\systemesnum}{}
\newcommand{\systemessansaccent}{}
\newcommand{\ilot}{2}
\newcommand{\ilotstr}{02}
\newcommand{\dossierilot}{\detokenize{Ilot_02 }}

