\input{../../headers/tdheaders.tex}

\section{Tracer un trait}

Le but de ce travail va être de faire tracer un trait \og à main levée\fg\ à Simone afin de vérifier son aptitude à se passer d'une règle.

\begin{figure}[h!]
\begin{center}
  \resizebox{0.5\textwidth}{!}{\input{img/vue_bras.pdf_tex}}
  \caption{\label{fig01} Bras paramétré}
\end{center}
\end{figure}


\question{Tracer le graphe des liaisons de cette sous-partie de Simone.}

\question{Justifier qu'il s'agit d'un cycle ouvert et déterminer son degré d'hyperstatisme.}

On définit $x$ et $y$ tels que $\overrightarrow{AC}=x\cdot\vec{x}+y\cdot\vec{y}$.

On cherche à tracer la droite $y(x)=c\cdot x+d$ reliant les points $\left(\begin{array}{c}a+b\\0
\end{array}\right)_R$ et $\left(\begin{array}{c}0\\a+b
\end{array}\right)_R$.

\question{Déterminer $c$ et $d$ en fonction de $a$ et $b$.} 

\question{Déterminer $x$ et $y$ en fonction de $\alpha$, de $\beta$ et des dimensions du système.} 

\question{Déterminer $x$ et $y$ en fonction de $\theta$ et de $u=\left\|\overrightarrow{AC}\right\|$.} 

\question{Déterminer $u=\left\|\overrightarrow{AC}\right\|$ en fonction de $\theta$ et des dimensions du système.} 

\question{Déterminer $\alpha$ et $\beta$ en fonction de $\theta$, de $u=\left\|\overrightarrow{AC}\right\|$ et des dimensions du système. Et justifier que:
\begin{center}
$\alpha=\theta\pm\arccos\left(\dfrac{u^2+a^2-b^2}{2\cdot u\cdot a}\right)$ et $\beta=\theta-\alpha\pm\arccos\left(\dfrac{u^2+b^2-a^2}{2\cdot u\cdot b}\right)$
\end{center}
} 

\question{Tester le résultat à l'aide d'un code python.}

\section{Faire des squats}

Le but de ce travail va être de faire faire des \og squat\fg\ à Simone.

\begin{figure}[ht!]
\begin{minipage}{0.45\linewidth}
\begin{center}
 \includegraphics[width=0.8\linewidth]{img/squat}
\end{center}
\caption{\label{fig02} Mouvement de Squat}
\end{minipage}
\hfill
\begin{minipage}{0.45\linewidth}
\begin{center}
 \includegraphics[width=0.9\linewidth]{img/structure_simone.png}
\end{center}
\caption{\label{fig03} Structure de Simone}
\end{minipage}
\end{figure}

\begin{figure}[h!]
\begin{center}
  \resizebox{0.4\textwidth}{!}{\input{img/squat_simone.pdf_tex}}
  \caption{\label{fig04} Bras paramétré}
\end{center}
\end{figure}

On notera respectivement $i_g$ et $i_d$ les pièces des jambes gauche et droite. On considérera que les deux pieds font partie de la classe équivalente \textit{sol}.

\question{Tracer le graphe des liaisons de cette sous-partie de Simone.}

\question{Justifier qu'il s'agit d'un cycle fermé et déterminer son degré d'hyperstatisme.}

\question{Proposer une modification d'une liaison pour rendre le système isostatique.}

\clearpage

\ifdef{\public}{\end{document}}{}

\newpage

\pagestyle{correction}\setcounter{section}{0}

\section{Correction}

\subsection{Tracer un trait}

\paragraph{Question 1:} ~\ \\

\begin{figure}[h!]
\begin{center}
  \resizebox{0.4\textwidth}{!}{\input{img/graphe_liaison_1.pdf_tex}}
  \caption{\label{fig05} Graphe de liaison 1}
\end{center}
\end{figure}

\paragraph{Question 2:} C'est un cycle ouvert car il n'y a pas de cycle fermé dans le graphe.

\paragraph{Question 3:}
 
$y(0)=d=a+b$, donc $d=a+b$.

$y(a+b)=c\cdot (a+b)+a+b=0$, donc $c=-1$

\paragraph{Question 4:}

$\overrightarrow{AC}=\overrightarrow{AB}+\overrightarrow{BC}$

$\overrightarrow{AC}=a\cdot\vec{x_1}+b\cdot\vec{x_2}$

$\overrightarrow{AC}=(a\cdot cos\alpha+b\cdot cos(\alpha+\beta))\vec{x_0}+(a\cdot sin\alpha+b\cdot sin(\alpha+\beta))\vec{y_0}$

Ainsi:
$\left\{\begin{array}{l}
x=a\cdot cos\alpha+b\cdot cos(\alpha+\beta)\\
y=a\cdot sin\alpha+b\cdot sin(\alpha+\beta)
\end{array}\right.$


\paragraph{Question 5:}

Ainsi:
$\left\{\begin{array}{l}
x=u\cdot cos\theta\\
y=u\cdot sin\theta
\end{array}\right.$

\paragraph{Question 6:} $u=\dfrac{a+b}{\sqrt{2}\cdot cos\left(\theta-\dfrac{\pi}{4}\right)}$

\begin{figure}[h!]
\begin{center}
  \resizebox{0.4\textwidth}{!}{\input{img/construction_u.pdf_tex}}
  \caption{\label{fig06} Construction pour u}
\end{center}
\end{figure}

\paragraph{Question 7:}

$\left\{\begin{array}{l}
u\cdot cos\theta=a\cdot cos\alpha+b\cdot cos(\alpha+\beta)\\
u\cdot sin\theta=a\cdot sin\alpha+b\cdot sin(\alpha+\beta)
\end{array}\right.$

$\left\{\begin{array}{l}
(b\cdot cos(\alpha+\beta))^2=(u\cdot cos\theta-a\cdot cos\alpha)^2\\
(b\cdot sin(\alpha+\beta))^2=(u\cdot sin\theta-a\cdot sin\alpha)^2
\end{array}\right.$

$b^2=(u\cdot cos\theta-a\cdot cos\alpha)^2+(u\cdot sin\theta-a\cdot sin\alpha)^2$

$b^2=(u\cdot cos\theta)^2+(a\cdot cos\alpha)^2-2\cdot u\cdot cos\theta\cdot a\cdot cos\alpha+(u\cdot sin\theta)^2+(a\cdot sin\alpha)^2-2\cdot u\cdot sin\theta\cdot a\cdot sin\alpha$

$2\cdot u\cdot a\cdot \left(sin\theta\cdot sin\alpha+cos\theta\cdot cos\alpha\right)=u^2+a^2-b^2$

$2\cdot u\cdot a\cdot cos(\theta-\alpha)=u^2+a^2-b^2$

$cos(\theta-\alpha)=\dfrac{u^2+a^2-b^2}{2\cdot u\cdot a}$

$\theta-\alpha=\pm\arccos\left((\dfrac{u^2+a^2-b^2}{2\cdot u\cdot a}\right)$

$\alpha=\theta\pm\arccos\left((\dfrac{u^2+a^2-b^2}{2\cdot u\cdot a}\right)$

On montre de même que:

$\left\{\begin{array}{l}
(a\cdot cos\alpha)^2=(u\cdot cos\theta-b\cdot cos(\alpha+\beta))^2\\
(a\cdot sin\alpha)^2=(u\cdot sin\theta-b\cdot sin(\alpha+\beta))^2
\end{array}\right.$

$a^2=(u\cdot cos\theta-b\cdot cos(\alpha+\beta))^2+(u\cdot sin\theta-b\cdot sin(\alpha+\beta))^2$

$a^2=u^2+b^2-2\cdot u\cdot cos\theta\cdot b\cdot cos(\alpha+\beta)-2\cdot u\cdot sin\theta\cdot b\cdot sin(\alpha+\beta)$

$a^2=u^2+b^2-2\cdot u\cdot b\cdot cos(\theta-(\alpha+\beta))$

$cos(\theta-(\alpha+\beta))=\dfrac{u^2+b^2-a^2}{2\cdot u\cdot b}$


$\beta=\theta-\alpha\pm\arccos\left(\dfrac{u^2+b^2-a^2}{2\cdot u\cdot b}\right)$

\paragraph{Question 8:}

\begin{minted}[xleftmargin=2em,linenos,firstnumber=1]{python}
import numpy as np
import matplotlib.pyplot as plt

a=45
b=65

theta=np.linspace(0,np.pi/2,100)
u=np.sqrt(2)/2*(a+b)/np.cos(theta-np.pi/4)
alpha=theta-np.arccos((u**2+a**2-b**2)/(2*u*a))
beta=theta-alpha+np.arccos((u**2-a**2+b**2)/(2*u*b))

x=a*np.cos(alpha)+b*np.cos(alpha+beta)
y=a*np.sin(alpha)+b*np.sin(alpha+beta)

plt.plot(x,y)
plt.show()
\end{minted}

\subsection{Faire des squats}

\paragraph{Question 1:} ~\ \\

\begin{figure}[h!]
\begin{center}
  \resizebox{0.7\textwidth}{!}{\input{img/graphe_liaison_2.pdf_tex}}
  \caption{\label{fig07} Graphe de liaison 2}
\end{center}
\end{figure}

\paragraph{Question 2:} 

\textit{Méthode statique:}

Il y a  6 liaisons pivot, donc $Ns=6\times 5=30$.

$rs=6\cdot (p-1)-m=6\times (6-1)-3=30-3=27$.

$h=3$

\textit{Méthode cinématique:}

Il y a  6 liaisons pivot, donc $Ic=6\times 1=6$.

Il n'y a qu'un cycle, donc $E=6$.

$h = m-Ic+E$

$h = 3-6+6=3$

\paragraph{Question 3:} On pourrait remplacer la liaison pivot d'axe $(A,\vec{z_0})$ par une linéaire annulaire d'axe $(A,\vec{z_0})$.

\end{document}
